\section[Procesos electroquimicos]{Procesos Electroqu\'{\i}micos}
En esta parte se retoman y aplican los conocimientos previamente adquiridos sobre  electr\'olitos, oxidaci\'on y reducci\'on, y  se relacionan con los procesos electroqu\'{\i}micos.  Se  estudia la transformaci\'on directa de energ\'{\i}a qu\'{\i}mica en energ\'{\i}a
el\'ectrica como un hecho cotidiano (acumuladores, pilas, etc).

Adem\'as se introduce el concepto de fuerza electromotriz (FEM) como medio para calcular la diferencia de potencial en las pilas y la espontaneidad de la corrosi\'on.

Finalmente se revisan los procesos de corrosi\'on, las condiciones que la favorecen, su
costo anual por deterioro de los metales y se discuten diferentes m\'etodos para prevenirla
o evitarla.
\subsection{Reacciones oxidaci\'on-reducci\'on}
La \textbf{oxidaci\'on-reducci\'on} (redox)\index{redox}, es un proceso  qu\'{\i}mico en el que cambia el n\'umero de oxidaci\'on de un elemento debido a una transferencia de electrones, en donde una substancia gana mientras otra substancia pierde..

Hay \textbf{oxidaci\'on} \index{oxidaci\'on}siempre que el n\'umero de  oxidaci\'on de un elemento aumente como resultado de p\'erdida de electrones. Al rev\'es, la \textbf{reducci\'on}  es cuando el n\'umero de oxidaci\'on de un elemento disminuye como resultado de ganancia de electrones. La oxidaci\'on y la reducci\'on se llevan a cabo simult\'aneamente en una reacci\'on qu\'{\i}mica; no se puede presentar una sin que se presente la otra.

El n\'umero o \'{\i}ndice de oxidaci\'on \index{oxidaci\'on!\'{\i}ndice}de un \'atomo puede considerarse que representa al n\'umero de electrones ganado, perdido o compartido desigualmente por dicho \'atomo. Cuando el n\'umero de oxidaci\'on es \textit{positivo}, el \'atomo tiene menos electrones asignados que los que hay en el \'atomo neutro. Cuando el n\'umero\index{oxidaci\'on!n\'umero} de oxidaci\'on es \textit{negativo}, el \'atomo tiene m\'as electrones asignados que los que tiene el citado \'atomo neutro. Cuando el n\'umero de oxidaci\'on  es \textit{cero}, el \'atomo tiene el mismo n\'umero de electrones asignados que los que tiene el \'atomo neutro.

\begin{table}[ht]
\caption{Reglas para asignar el n\'umero de oxidaci\'on.}
\begin{center}
\begin{tabular}{ cl }\hline
\textbf{1.}& Todos lo elementos en estado libre (no combinados con otros ele-\\
&mentos) tiene un n\'umero de oxidaci\'on igual a
 cero (por ejemplo,\\
&Na, Cu, Mg, H$_2$, O$_2$, Cl$_2$, N$_2$).\\
\textbf{2.}& Para el H es +1.\\
\textbf{3.}& El n\'umero de oxidaci\'on del ox\'{\i}geno O es $-2$.\\
\textbf{4.}& El elemento met\'alico en un compuesto i\'onico tiene n\'umero de\\
&oxidaci\'on positivo.\\
\textbf{5.}& La suma algebraica de los n\'umeros de oxidaci\'on de los elementos\\ &en un
compuesto es igual a cero.\\
\textbf{6.}& La suma algebraica de los n\'umeros de oxidaci\'on de los elementos\\ &en un
i\'on poliat\'omico es igual a la carga del i\'on.\\ \hline
\end{tabular}
\end{center}
\label{tab:5}
\end{table}

En la mol\'ecula de agua\index{agua!mol\'ecula} (\ce{H2O}) , al \'atomo de ox\'{\i}geno 
\textbf{O} se le asigna un n\'umero de oxidaci\'on de $-2$ y a cada \'atomo de hidr\'ogeno
\textbf{H} se le asigna un n\'umero de oxidaci\'on de $+1$.

 En el amon\'{\i}aco \ce{NH3}. A cada \'atomo de hidr\'ogeno
\textbf{H} se le asigna un n\'umero de oxidaci\'on de $+1$, por lo tanto al nitr\'ogeno
\textbf{N}, se le asigna un n\'umero de oxidaci\'on de $-3$.

Para el carbonato \textbf{CO}$^{2-}_3$ que es un i\'on, la suma de los \'{\i}ndices de
oxidaci\'on de los \'atomos de carbono \textbf{C} y de ox\'{\i}geno \textbf{O} debe ser $2-$,  
que es la carga del i\'on. El n\'umero de oxidaci\'on de cada \'atomo de
ox\'{\i}geno \textbf{O} es $-2$ entonces:
 $$q_{\textrm{carbono}} + 3(-2) = -2$$
$$q_{\textrm{carbono}} = -2 + 6 = 4$$
Los n\'umeros de oxidaci\'on son para el carbono \textbf{C}, $+4$; y del ox\'{\i}geno
\textbf{O},$-2$

\begin{table}[hbt]
\caption[N\'umeros de oxidaci\'on]{N\'umeros de oxidaci\'on de los \'atomos en algunos
compuestos}
\begin{center}
{\footnotesize \begin{tabular}{l|rll}\hline 
\textbf{I\'on o compuesto}&\multicolumn{3}{c}{\textbf{N\'umero de oxidaci\'on}}\\ \hline
 H$_2$O &H, $+1$; & O, $-2$   \\ SO$_2$ &S, $+4$; & O, $-2$   \\
CH$_4$ &C, $+4$; & H, $-1$   \\
KMnO$_4$& K, $+1$;& Mn,$+7$;&O,$-2$   \\
Na$_3$PO$_4$ & Na, $+1$;& P, $+7$;& O,$-2$ \\
Al$_2$(SO$_4$)$_3$ &Al,$+3$;&S,$+6$&O,$-2$\\
NO  & N,  $+2$; & O, $-2$   \\
BCl$_3$ &B, $+3$ & Cl, $-1$\\
Fe$_2$O$_3$ &Fe, $+3$ & O, $-2$   \\
SO$_4^{2-}$ &S, $+6$; & O, $-2$  \\
NO$_3^-$  & N,  $+5$; & O, $-2$  \\
CO$_3^{2-}$  & C,  $+4$; & O, $-2$  \\ \hline
\end{tabular}}
\end{center}
\label{tab:6}
\end{table}\index{numero@n\'umero de oxidaci\'on}
La sustancia que causa un aumento en el estado de oxidaci\'on de
otra se llama \textbf{agente oxidante}\index{agente!oxidante}. La sustancia que causa una
disminuci\'on del estado de oxidaci\'on de otra se denomina
\textbf{agente reductor}\index{agente!reductor}.

La oxidaci\'on-reducci\'on sucede en muchas reacciones de combinaci\'on, des\-com\-posici\'on y de desplazamiento sencillo. Por ejemplo:

\begin{equation}
\ce{2H2 +  O2 -> 2H2O} 
\end{equation}

Ambos reactivos, el hidr\'ogeno y el ox\'{\i}geno, son elementos en estado libre y tienen un n\'umero de oxidaci\'on de cero. En el producto, agua, el hidr\'ogeno se ha oxidado a $+1$, y el ox\'{\i}geno se ha reducido a $-2$.

Reacciones de oxidaci\'on\index{reaccion@reacci\'on!oxidaci\'on}:

\begin{eqnarray}
\ce{ 2Fe^{\circ} +O2 ->} & \ce{2FeO}\label{eqfe:1}\\
\ce{ 2Fe^{\circ} + 3/2O2 ->}&\ce{Fe2O3}\label{eqfe:2}
\end{eqnarray}

En estas reacciones se observa que el hierro pasa de su estado natural a uno de oxidaci\'on para el primer caso reacci\'on (\ref{eqfe:1}) dos \'atomos de hierro reaccionan con dos \'atomos de ox\'{\i}geno, si el ox\'{\i}geno \textbf{O}  \index{oxigeno@ox\'{\i}geno!numero@n\'umero de oxidaci\'on}tiene un n\'umero de oxidaci\'on de 2, entonces para el hierro\index{hierro!numero@n\'umero de oxidaci\'on} \textbf{Fe} su n\'umero de oxidaci\'on ser\'a $+$2. Para la segunda reacci\'on (\ref{eqfe:2}), tenemos 3 \'atomos de ox\'{\i}geno \textbf{O} con lo que nos da  $3\times-2=-6$ y como hay 2 \'atomos de hierro \textbf{Fe} el n\'umero de oxidaci\'on de  \'este ultimo es $6/2=3$. Estos n\'umeros de oxidaci\'on los podemos ver en la \textbf{Cuadro~\ref{tab:6}} de la p\'agina \pageref{tab:6}.


\subsection{Celdas electroqu\'{\i}micas}\index{celda}
\paragraph{Celdas galv\'anicas o voltaicas}\index{celda!voltaica}
En estas celdas las reacciones electro\-qu\'{\i}\-mi\-cas convierten
la energ\'{\i}a  qu\'{\i}mica en el\'ectrica de manera espont\'anea y
el cambio qu\'{\i}mico produce electricidad. Los sistemas
electroqu\'{\i}micos en donde ocurren estas reacciones se llaman
tambi\'en celdas elec\-tro\-qu\'{\i}\-mi\-cas. Por ejemplo la pila seca \index{pila!seca} es
una celda galv\'anica, electroqu\'{\i}mica o voltaica, como en la Figura \ref{fig2:1}.

\paragraph{Celdas electrol\'{\i}ticas} \index{celda!electrolitica@electrol\'{\i}tica} En estas celdas las reacciones elec\-tro\-qu\'{\i}\-mi\-cas no son
espont\'aneas, para ocurrir requieren de electricidad. La energ\'{\i}a e\-l\'ec\-tri\-ca induce a que ocurra la reacci\'on
qu\'{\i}mica no espont\'anea. El cambio qu\'{\i}\-mi\-co producido por la corriente el\'ectrica se llama electr\'olisis. Los sistemas
electroqu\'{\i}micos donde ocurren estas reacciones se llaman celdas electro\-l\'{\i}\-ti\-cas.
\begin{figure}
%\vglue 3in
%\hspace {.5in}\special{picture Celda}
%\vglue 0.25in

\hspace{0.15in} \resizebox{10cm}{7cm}{\includegraphics{figuras/daniel.eps}}
%\begin{picture}(90,65)(-20,0)
%\linethickness{.8mm}
%\multiput( 5,10)(30,0){2}{\line(0,1){40}}
%\multiput(55,10)(30,0){2}{\line(0,1){40}}
%\multiput(10, 5)(50,0){2}{\line(1,0){20}}
%\qbezier(10, 5)( 5, 5)(4.5,10)
%\qbezier(30, 5)(35, 5)(35,10)
%\qbezier(60, 5)(55, 5)(55,10)
%\qbezier(80, 5)(85, 5)(85,10)
%%Puente salino
%\linethickness{.5mm}
%\multiput(23,25)(44,0){2}{\line(0,1){30}}
%\multiput(30,25)(30,0){2}{\line(0,1){25}}
%\put(35,55){\line(1,0){20}}
%\put(30,62){\line(1,0){30}}
%\qbezier(30,50)(30,55)(35,55)
%\qbezier(60,50)(60,55)(55,55)
%\qbezier(23,55)(23,62)(30,62)
%\qbezier(60,62)(67,62)(67,55)
%\linethickness{.2mm}
%\dottedline{.5}(5,40)(35,40)
%\dottedline{.5}(55,40)(85,40)
%\end{picture}
\caption{Celda electroqu\'{\i}mica}
\label{fig2:1}
\end{figure}
Toda celda galv\'anica o electrol\'{\i}tica, se compone de las siguientes par\-tes:

\begin{description}
\item [Celda] Consta de un \'anodo \index{anodo@\'anodo}y un c\'atodo \index{c\'atodo}
sumergidos en un electr\'olito co\-m\'un y conectados entre
s\'{\i}.  La celda puede estar en su propio recipiente o ser el compartimiento individual de
una pila. La celda completa es el sistema electroqu\'{\i}mico.
\item[Cuba] \index{cuba} Recipiente que contiene la disoluci\'on
electrol\'{\i}tica y los electrodos.
\item [Disoluci\'on electrol\'{\i}tica] La mezcla homog\'enea de un
electr\'olito en agua.
\item[Electr\'olito] \index{electrolito} Sustancia soluble en agua y formadora de iones
que por lo mismo conduce la corriente el\'ectrica. Por ejemplo
sales de CuSO$_4$, NaCl, \'acidos como el sulf\'urico 
(H$_2$SO$_4$), ac\'etico (CH$_3$COOH), bases como el NaOH, KOH, etc.
\item[Electrodo] \index{electrodo}Cuerpo que intercambia electrones entre el
circuito el\'ectrico y la disoluci\'on. F\'{\i}sicamente puede ser
una placa met\'alica, una barra de alg\'un material inerte, como
el platino o un sistema que burbujea gas, como el llamado electrodo
\index{electrodo!hidr\'ogeno} de hidr\'ogeno.
\item[Circuito el\'ectrico] Un simple conductor met\'alico que une
a los electrodos externamente al sistema electroqu\'{\i}mico.
\end{description}

\subsection{Sistemas electroqu\'{\i}micos}
Una clasificaci\'on de los sistemas electroqu\'{\i}micos es la siguiente:
\begin{description}
\item[Celdas electroqu\'{\i}micas]
Galv\'anicas o volt\'aicas\index{celda!electroquimica@electroqu\'{\i}mica}
\begin{itemize}
 \item Primarias (irreversibles) \index{pila}
  Pilas secas, h\'umedas, de reserva de e\-lec\-tr\'olito s\'olido.
\index{celda!electroquimica@electroqu\'{\i}mica!primaria}
\item Secundarias (reversibles)
  Acumuladores: \'acidos,
b\'asicos.\index{celda!electroquimica@electroqu\'{\i}mica!reversible}
\end{itemize}

\item[Celdas electrol\'{\i}ticas] Donde los procesos de ruptura de enlaces se da mediante
el suministro de energ\'{\i}a el\'ectrica.
\index{celda!electrolitica@electrol\'{\i}tica}
\end{description}

\subsection{Potencial Est\'andar de Reducci\'on}
\index{potencial de reducci\'on}
La corriente \textit{el\'ectrica} es el flujo de  electrones que ocurre cuando hay una diferencia de potencial entre dos puntos unidos por un conductor met\'alico. De manera an\'aloga existe una diferencia de potencial el\'ectrico entre dos celdas  lo cual provoca una corriente el\'ectrica.

La diferencia de energ\'{\i}a potencial el\'ectrica entre dos semiceldas \index{celda!semicelda} se mide en volt (V). Esta diferencia de potencial el\'ectrico entre las dos medias celdas se les llamaba fuerza electromotriz (\gloss[Word]{fem}) en la actualidad se prefiere llamarle \textit{potencial de la celda}. El potencial de la celda es la fuerza motora que empuja a los electrones a trav\'es del conductor, acoplado a la celda electroqu\'{\i}mica, desde un electrodo al otro por lo mismo tambi\'en mueve a la reacci\'on electroqu\'{\i}mica.

Un potencial de celda positivo expresa que la reacci\'on ocurre espont\'a\-nea\-mente, en cambio un valor negativo indica una reacci\'on no es\-pon\-t\'anea.
\begin{table}[ht]
\caption{Potenciales de reducci\'on}
\begin{center}
{\footnotesize \begin{tabular}{|l|l|r|}\hline \hline
\textbf{Semirreacci\'on}& &\textbf{E}$^\circ$,\\ 
Reactivos &Productos&volt\\ \hline \hline
\ce{F2 (g) + 2 e^-}& \ce{2F^-} & 2.87\\
\ce{H2O2 + 2H+ + 2e^-} & \ce{2H2O (l)}&1.76 \\
\ce{MnO4^- + 8H+ + 5e^-}  & \ce{Mn^2+ + 4H2O(l)} & 1.51\\
\ce{Cr2O7^2- + 14H+ + 6e^-} & \ce{2Cr^3+3 +7H2O(l)}&1.36\\
\ce{Cl2 (g) + 2e^-} & \ce{2Cl^-}  & 1.36 \\ 
\ce{O2 (g) + 4H+ + 4e^-}& \ce{2H2O(l)} &1.23 \\
\ce{Br2 (l) + 2e^-}& \ce{2Br^-} & 1.07 \\
\ce{Ag+ + e^-}&\ce{Ag (s)}  & 0.80\\
\ce{Fe^3+ + e^-} & \ce{Fe^2+} & 0.77 \\
\ce{I2 (s) + 2e^-} & \ce{2I^-} &0.54\\
\ce{Cu+ + e^-} & \ce{Cu (s)} &0.52 \\
\ce{Cu^2+  + 2e^-} & \ce{Cu (s)} & 0.34 \\ 
\ce{2H+ + 2e^-} &  \ce{H2 (g)}  & 0.0 \\
\ce{Fe^3+ + 3e^- }& \ce{Fe  (s)}  &   -0.04 \\
\ce{Fe^2+ + 2e^- }& \ce{Fe (s)} &   -0.44 \\
\ce{Zn^2+ + 2e^-} & \ce{Zn (s)}  &   -0.76 \\
\ce{Cr^2+ + 2e^-} & \ce{Cr (s)} &   -0.90\\
\ce{Al^3+ + 3e^-}& \ce{Al (s)}  &   -1.76\\
\ce{Mg^2+ + 2e^-} &\ce{Mg (s)} &   -2.36 \\
\ce{Na+ + e^-}     &\ce{Na (s)} &   -2.71 \\
\ce{Ca^2+ + 2e^-} &\ce{Ca (s)} &   -2.84 \\
\ce{Ba^2+ + 2e^-} & \ce{Ba (s)} &   -2.92 \\
\ce{Cs+ + e^-}     & \ce{Cs (s)} &   -2.92 \\
\ce{K+ + e^-}       & \ce{K (s)} &   -2.93 \\
\ce{Li+ + e^-}     & \ce{Li (s)} &   -3.05 \\ \hline
\end{tabular}}
\end{center}
\label{tabla7}
\end{table}

Consideremos una celda electroqu\'{\i}mica con dos
electrodos.\index{celdas!electroquimicas@electroqu\'{\i}micas!electrodos}  El potencial de
la celda  estar\'a definido por:
\begin{equation}
E_{\textrm{\tiny celda}} = E_{\textrm{\tiny reducci\'on}}-E_{\textrm{\tiny oxidaci\'on}}
\label{eq:28}
\end{equation}
\gloss[nocite]{ecelda}\gloss[nocite]{ereducc}\gloss[nocite]{eoxida}
\begin{example}
Si se tiene la siguiente celda

\begin{center}
\ce{Pt}$|$\ce{H2} (1 atm) $|$ \ce{H^+ (ac)} (1 M) $||$ \ce{Cu^2+ (ac)} (1M) $|$ \ce{Cu (s)}
\end{center}
Aplicando la ecuaci\'on \ref{eq:28} tenemos:
\begin{center}
$E_{\textrm{\tiny celda}} = E_{\ce{Cu^2+}|\ce{Cu}}-E_{\ce{H+}|\ce{H2}}$
\end{center}
A partir de los datos del \textbf{Cuadro~\ref{tabla7}} de la p\'agina \pageref{tabla7} obtenemos:
\begin{center}
$E_{\textrm{\tiny celda}} = 0.34  V - 0 V =  0.34 V$
\end{center}
\textit{\textbf{Recordar} que el potencial del electrodo de hidr\'ogeno \index{electrodo!hidr\'ogeno}es cero
por convenci\'on.}
\end{example}


En lugar de  dibujar un esquema como el de la \textbf{Figura~\ref{fig2:1}} de la
p\'agina \pageref{fig2:1} es m\'as sencillo hacer un diagrama de celda para describirla. El
diagrama de
\index{celda!diagrama} celda para la pila de Daniell \index{pila!Daniell} es:

$$\ce{Zn(s)}|\ce{Zn^2+}(ac)||\ce{Cu^2+}(ac)|\ce{Cu(s)}$$

Observa que para escribir el diagrama de celda el \'anodo siempre se escribe a la izquierda
y el c\'atodo hasta la extrema derecha. La l\'{\i}nea vertical `$|$' representa la frontera
entre las fases como ocurre entre el Zn de la placa y la disoluci\'on de sus iones. La doble
l\'{\i}nea vertical `$||$' denota el puente salino.
\begin{center}
Iones provenientes de la sal disuelta\\
$\swarrow \searrow$\\
\'Anodo \ce{Zn(s)}$|$\ce{Zn^2+(ac)} $||$ \ce{Cu^2+ (ac)} $|$ \ce{Cu(s)}  C\'atodo
\end{center}
\begin{description}
\item[Oxidaci\'on] \ce{Zn -> Zn^2+ + 2 e^-}
\item[Reducci\'on] \ce{Cu^{2+} + 2e^- -> Cu}
\end{description}
Aplicando la Ecuaci\'on~\ref{eq:28} en de la p\'agina~\pageref{eq:28} tenemos:

$$E_{\textrm{\tiny celda}} = E_{\ce{Cu^2+}|\ce{Cu}}-E_{\ce{Zn^2+}|\ce{Zn (s)}}$$

A partir de los datos del \textbf{Cuadro~\ref{tabla7}} de la p\'agina~\pageref{tabla7} obtenemos:
$$E_{\textrm{\tiny celda}} = 0.34  V - (-0.76 V) =  1.10 V$$
\begin{example}
Calcular el potencial de la celda:
$$\ce{Al (s)}|\ce{Al^3+ (ac)}||\ce{Fe^3+ (ac)}| \ce{Fe (s)}$$
Aplicando la Ecuaci\'on~\ref{eq:28} de la p\'agina~\pageref{eq:28} tenemos:
$$E_{\textrm{\tiny celda}} = E_{\ce{Fe^3+}|\ce{Fe}}-E_{\ce{Al^3+}|\ce{Al (s)}}$$
A partir de los datos del \textbf{Cuadro~\ref{tabla7}} de la p\'agina~\pageref{tabla7} obtenemos:
$$E_{\textrm{\tiny celda}} = -0.04  V - (-1.76 V) =  1.72 V$$
Como es \textbf{positivo} el valor del potencial, esta reacci\'on es
espont\'anea.
\end{example}
\begin{example}
Calcular el potencial de la celda:
$$\ce{Ag (s)}| \ce{Ag+ (ac)}|| \ce{Na+ (ac)}| \ce{Na (s)}$$
Aplicando la ecuaci\'on \ref{eq:28} tenemos:
$$E_{\textrm{\tiny celda}} =E_{\ce{Na+}|\ce{Na (s)}}- E_{\ce{Ag+}|\ce{Ag (s)}}$$
A partir de los datos del \textbf{Cuadro~\ref{tabla7}} de la p\'agina~\pageref{tabla7} obtenemos:
$$E_{\textrm{\tiny celda}} = -2.71  V - (0.80 V) = -3.51 V$$
Como es \textbf{negativo} el valor del potencial, esta reacci\'on \textbf{no} es espont\'anea.
\end{example}


\subsection[Corrosi\'on de metales]{Corrosi\'on de metales, un proceso espont\'aneo}
\index{corrosion@corrosi\'on}
Como hemos visto la oxidaci\'on del hierro es un proceso exot\'ermico que ocurre con aumento de entrop\'{\i}a, y cuya $\Delta G$ de reacci\'on es negativa por lo cual podemos decir que es un  reacci\'on espont\'anea. Ahora electroqu\'{\i}mica\-men\-te hablando la reacci\'on de oxidaci\'on del fierro se da debido a que este material se oxida mientras alg\'un otro se esta reduciendo. 

Como podemos observar en la \textbf{Cuadro~\ref{tabla7}} de la p\'agina~\pageref{tabla7}, los potenciales de reducci\'on del Fe se encuentra en un nivel inferior por lo que tender\'a a oxidarse m\'as f\'acilmente que a reducirse. Esto causa que al calcular el potencial de celda del hierro con cualquier otro compuesto nos d\'e positivo indicando que la reacci\'on es posible.

\subsection{Prevenci\'on de la corrosi\'on}

La corrosi\'on \index{corrosion@corrosi\'on!prevencion@prevenci\'on} es un proceso
electroqu\'{\i}mico en el cual existen dos celdas y un medio conductor, el \'anodo se
corroe ya que aporta electrones mientras que el c\'atodo se protege.
Existen diferentes m\'etodos con los cuales se puede reducir la corrosi\'on de materiales
que nos interesa proteger, as\'{\i} tenemos los siguientes:

\begin{description}
\item[a)]\'Anodo de sacrificio\index{anodo@\'anodo!de sacrificio}. Se pueden adherir placas de metales que se o\-xidan m\'as f\'acilmente (magnesio, zinc, titanio) sobre los materiales que se desean proteger (hierro). Tambi\'en se emplean recubrimientos an\'odicos  (como el galvanizado). A una l\'amina de hierro se le recubre con zinc par evitar la corrosi\'on del hierro ya que el zinc se oxida mientras el hierro se protege.
\index{corrosion@corrosi\'on!prevencion@prevenci\'on!anodo@\'anodo de sacrificio}
\item[b)]Aislamiento. Al material que se desea proteger se le hace un recubri\-miento con pinturas, en otros casos se a\'{\i}sla un materiales de composici\'on  diferente para evitar el par galv\'anico. Una tuber\'{\i}a de hierro de una de cobre.
\index{corrosion@corrosi\'on!prevencion@prevenci\'on!aislamiento}
\item[c)] Rectificadores.\index{rectificador} Se conecta el material a proteger a un banco de \'anodos donde se regula el potencial de protecci\'on entre el material y el suelo.
\index{corrosion@corrosi\'on!prevencion@prevenci\'on!rectificadores}
\end{description}
En t\'erminos generales se desea evitar forma un \gloss[word]{pargalvanico}  \index{par galv\'anico}ya sea que no entren en contacto el \'anodo y el c\'atodo, o si no es posible se transfiera la oxidaci\'on de un material a otro como en el caso de los \'anodos de sacrificio.
