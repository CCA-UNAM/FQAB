\chapter{La atm\'osfera}
 
 La capa gaseosa que rodea la Tierra se le conoce como atm\'osfera, los planetas peque\'~nos poseen menos por que tiene menos masa.
 \section{Estrucutra de la Atm\'osfera}
 Las capas de la atmosfera son las siguientes:
 \begin{description}
\item[Exosfera ] se encuentra ubicada a 500\kilo\metre\, posee \'atomos y mol\'eculas no unidas por la gravedad.
\item[Term\'osfera] entre 400 a 500\kilo\metre, $10^6$ -- $10^{14}$ menos denso que en la superficie. 1,000 -- 2,000\kelvin. Se encuentran \'atomos de \ce{O2}, \ce{N2}. La qu\'{\i}mica se rige por la luz ultravioleta $\lambda\leqq 150$nm. $\alpha,\beta$ 121.6$\nano\metre$ 102.6\nano\metre. En esta capa se encuentran  rayos X.
\item[Mesosfera] de 50-100$\kilo\metre$  ($\pm10$). La concentraci\'on de mol\'eculas va de 10$^{13}$--10$^{16}$ molec/cc. La temperatura va de los 130 -- 250 \kelvin. L qu\'{\i}mica es dirigida por luz ultravioleta $\lambda=250\nano\metre$. Se encuentran mol\'eculas simples \ce{O2}, \ce{N2}, \ce{CO2}, \ce{H2O}, \ce{O3}, radicales como \ce{OH}, \ce{HO2} iones \ce{O2-},\ce{NO2-}
\item[Ionosfera]Sobre 60 \kilo\metre. Se encuentran electrones libres cuya concentraci\'on puede alcanzar $10^6$molec/cc. En esta capa se da la propagaci\'on de las ondas de radio
\item[Estratosfera] Como su nombre lo indica esta en capas de 10--17 \kilo\metre\~hasta 50--50\kilo\metre. El mezclado vertical toma a\~nos. La qu\'{\i}mica es dirigida por luz ultravioleta, visible con longitudes de onda mayores a los 175\nano\metre. Existen mol\'eculas mas complejas \ce{HNO3}, \ce{HO2}, \ce{NO2}, \ce{CH3O2}, \ce{ClONO2}, etc. Posee algunas nubes.
\item[Troposfera] De la superficie a 10--17\kilo\metre. El mezclado vertical es r\'apido de minutos a d\'{\i}as. L temperatura oscila de 200-300 K. Hay mol\'eculas complejas hasta de cerca de \ce{C12H26} en fase gaseosa.
\end{description}

\section{El ozono } \index{ozono}
 \label{esozono}
El ozono (\ce{O3}) es una forma alotr\'opica del ox\'{\i}geno (\ce{O2}) que esta constituido por tres \'atomos de ox\'{\i}geno. Se encuentra en la estratosfera, a una altura de entre 20 y 30 \kilo\metre , donde forma la capa de ozono.

Esta capa absorbe a los rayos ultravioleta del Sol. Si esa acci\'on, los rayos solares podr\'{\i}an causar c\'ancer de piel, mutaciones y deficiencias inmunitarias. 

El ozono estratosf\'erico se ve afectado por los cloro fluorocarbonos (CFC). Estos compuestos son inertes en la troposfera y debido a ello se desplazan hacia la estrat\'osfera. En el polo Sur mediante procesos catal\'{\i}ticos 


 