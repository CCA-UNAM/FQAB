\chapter*{Introducci\'on}

Dentro del posgrado en Ciencias de la Tierra se recibe una gran diversidad de estudiantes que provienen de diferentes \'areas del conocimiento. Si bien muchos de ellos estudiaron y revisaron durante el bachillerato parte de los temas que aqu\'{\i} se presentan, \'estos no son revisados dentro de sus respectivas carreras profesionales. El motivo de este trabajo es ayudar  a recordar y a presentar los temas relacionados a la fisico qu\'{\i}mica atmosf\'erica. que sean \'utiles para el desarrollo de sus estudios en lel posgrado.

 En la primera parte se presentan los temas fundamentales de la termodin\'a mica, con base en las leyes de la termodin\'a mica, energ\'{\i}a libre, reacciones en los seres vivos. Posteriormente se abordan los temas de procesos electroqu\'{\i}micos donde se ven los fen\'omenos de \'oxido-reducci\'on, los sistemas electroqu\'{\i}micos y la corrosi\'on.
 
 Si bien la termodin\'amica nos indica la espontaneidad de un proceso, lo anterior no nos indica la rapidez con la que se puede dar, por lo que se abordan los temas de cin\'etica qu\'{\i}mica, teor\'{\i}a de colisiones, energ\'{\i}a de activaci\'on  y los factores que influyen en la rapidez de una reacci\'on.
 
 En la parte de equilibrio qu\'{\i}mico se aborda el tema reversibilidad de las reacciones, la constante de equilibrio y algunas aplicaciones de la misma con el $p$H.
 
 En la parte de qu\'{\i}mica org\'anica se hace una descripci\'on de los niveles de energ\'{\i}a en los \'atomos, las cofiguraciones electr\'onicas, los s\'{\i}mbolos de Lewis, lo que es la energ\'{\i}a de ionizaci\'on. Se muestran las familias de compuestos org\'anicos basados en el n\'umero de enlaces entre \'atomos de carbono y con otros elementos como el ox\'{\i}geo y el nitr\'ogeno.
 
 En la segunda parte se muestran los compuestos considerados como contaminantes y su clasificaci\'on, el manejo de unidades de los mismos y los procesos fisicoquimos  en la atm\'esfera en los que \'estos intervienen.
 
 Si el lector desea profundizar en  alg�n tema se presenta la bibliograf\'{\i}a complementaria a los temas tratados.
 
 Se espera que sea un libro texto para aquellos quienes los temas de qu\'{\i}mica son poco estudiados dentro de sus formaci\'on profesional.
 
 