\chapter[Energia y reacciones]{La energ\'{\i}a y las reacciones qu\'{\i}micas}

El estudio de la energía y las reacciones químicas es fundamental para comprender los procesos que rigen tanto la naturaleza como las aplicaciones tecnológicas. Este capítulo explora los principios básicos de la termodinámica, la electroquímica y la termoquímica, proporcionando una base sólida para entender cómo la energía se transforma y se transfiere en los sistemas químicos.

En la \textbf{Sección 1.1}, se introduce el concepto de \textbf{energía} y su relación con las reacciones químicas. Esta parte se inicia con el desarrollo de la noción de sistema, estados y funciones de estado. Se analizan los tipos de sistemas, las propiedades de un sistema y los cambios de estado, junto con el comportamiento de los gases ideales y no ideales, las ecuaciones de estado y la Ley General de los Gases. Además, se interpreta el significado de la primera ley de la termodinámica, expresada como $\Delta E = q + w$, que establece la conservación de la energía. Se establece la diferencia entre energía interna y entalpía, y se clasifican las reacciones en exotérmicas y endotérmicas.

En la \textbf{Sección 1.2}, se estudia la forma en que se presentan gráficamente los cambios de energía asociados a los cambios químicos, así como la relación entre las entalpías de reacción y las energías de enlace. Se definen los estados estándares y se explica cómo las entalpías de formación y la Ley de Hess se aplican en cálculos termoquímicos. Estos conceptos permiten comprender y predecir los cambios energéticos en reacciones químicas complejas.

En la \textbf{Sección 1.3}, se introduce el concepto de \textbf{entropía} ($\Delta S$), una medida del desorden en un sistema, y su relación con los cambios de energía en transformaciones físicas y químicas. Se interpreta y aplica la ecuación que relaciona energía libre, entalpía y entropía: $\Delta G = \Delta H - T \Delta S$. Se analiza la importancia de la variación de la energía libre durante un cambio químico como criterio para predecir, sin necesidad de experimentos, si un proceso puede ocurrir o no. Finalmente, se estudia la relación entre la energía libre, los procesos espontáneos y las reacciones exergónicas y endergónicas.

La \textbf{Sección 1.4} aborda los \textbf{procesos electroquímicos}, incluyendo las reacciones de oxidación-reducción, las celdas electroquímicas (galvánicas y electrolíticas), los sistemas electroquímicos y el potencial estándar de reducción. También se discuten temas relacionados con la corrosión de metales y las estrategias para su prevención.

Este capítulo proporciona una visión integral de los principios energéticos y químicos que rigen los procesos naturales y tecnológicos, sentando las bases para su aplicación en diversos campos de la ciencia y la ingeniería.

\section{Sistemas, estados y funciones de estado}
\index{sistema}
 Un \textit{\gloss[word]{sistema} termodin\'a\-mi\-co} es una porci\'on del universo separada por fronteras que nos interesa estudiar, tambi\'en se puede definir como la parte del universo f\'{\i}sico cuyas propiedades se est\'an investigando. El sistema est\'a confinado a un lugar definido en el espacio por la \gloss[word]{frontera}\index{sistema!frontera} que lo separa del resto del universo (el medio exterior). La ciencia que estudia c\'omo la energ\'{\i}a y propiedades mesurables de un sistema se relacionan y la interacci\'on con los alrededores es la \textbf{\gloss[word]{termodinamica}} \index{termodinamica@termodin\'amica} su nombre procede del griego \textit{termos} (calor) y \textit{dinamos} (movimiento). Es parte de la f\'{\i}sica que trata del estudio de los procesos en los que se transfiere energ\'{\i}a o sus manifestaciones como calor y como trabajo. 

\subsection{Tipos de sistemas}

\index{sistema!tipos}
\begin{itemize}
\item \textit{Aislado} cuando la frontera evita cualquier interacci\'on con la materia o  energ\'{\i}a del exterior.
\item \textit{Abierto} cuando pasa materia a trav\'es de la frontera.
\item \textit{Cerrado} aquel que no intercambia materia con sus alrededores pero s\'{\i} ener\-g\'{\i}a.
\item \textit{Homog\'eneo} cuando contiene \'unicamente una fase.
\item \textit{Heterog\'eneo} contiene m\'as de una fase.
\end{itemize}

Se define a una \gloss[word]{fase}\index{sistema!fase} como una porci\'on homog\'enea del sistema, f\'{\i}sica\-mente diferenciable y separable mec\'anicamente. Si el sistema ``agua'' coexisten las tres formas s\'olida (hielo), l\'{\i}quida y gaseosa (vapor), cada una constituye una fase separada de las otras por un l\'{\i}mite.


\subsection{Propiedades de un sistema}
Las propiedades de un sistema\index{sistema!propiedades} son los atributos f\'{\i}sicos que perciben los sentidos directamente o con un m\'etodo experimental. La presi\'on y el vo\-lumen son propiedades f\'acilmente perceptibles a las que se les designa un valor num\'erico. La entalp\'{\i}a es una propiedad que s\'olo se detecta con un experimento y no se mide su valor absoluto sino sus cambios.

Las propiedades del sistema se dividen en dos clases:

\begin{itemize}
\item \textit{Intensivas} \index{propiedades!intensivas} son aquellas que no dependen de la cantidad de substancia o del tama\~no de un sistema, por lo que el valor permanece inalterable al subdividir el sistema inicial en varios subsistemas. Entre estas propiedades se encuentra la densidad (\gloss[word]{densidad}), el \'{\i}ndice de refracci\'on, la masa molar, el volumen molar y la e\-nerg\'{\i}a molar. 
\item \textit{Extensivas} \index{propiedades!extensivas}son aquellas que si dependen de la cantidad de sustancia o del tama\~no de un sistema, son magnitudes cuyo valor es proporcional al tama\~no del sistema que describe, dependen de dos de las tres variables presi\'on ($P$), volumen ($V$) y temperatura ($T$) y el n\'umero de moles.
\end{itemize}

\subsection{Estado de un sistema}
\index{sistema!estado}
Un sistema se encuentra en un estado definido cuando cada una de sus propiedades tiene un valor determinado. Debemos saber que con base en el estudio experimental del sistema o de la experiencia con sistemas semejantes, qu\'e propiedades deben considerarse, con el fin de que el estado del sistema sea definido con precisi\'on suficiente para el prop\'osito buscado.

\begin{example}[Estado de un sistema]El estado est\'andar del agua l\'{\i}quida a $25\celsius$ y 1 \bbar \footnote{ La IUPAQ con base en el trabajo del Comit\'e de Datos para la ciencia y Tecnolog\'{\i}a (CODATA) recientemente recomend\'o que el estado est\'andar se refiera a la presi\'on de 1 \bbar (100 000\pascal). La diferencia entre una atm\'osfera y un bar es de alrededor de 1\% .}.
\end{example}

 El estado est\'andar se describe con un cero como super\'{\i}ndice, por ejemplo $\Delta H^\circ$. Este s\'{\i}mbolo expresa en los textos que siguen la convenci\'on antigua, que los valores corresponden a  la presi\'on de 1 atm\'osfera.

\subsection{Cambio de estado, trayectoria, ciclo, proceso}

Sometamos a un sistema a un cambio de estado \index{sistema!cambio de estado} desde
condiciones espec\'{\i}ficas iniciales hasta condiciones espec\'{\i}ficas finales.

El \textit{cambio de estado}\index{estado!cambio} est\'a completamente  definido cuando se especifica el estado inicial y el estado final.

La \textit{trayectoria} \index{estado!trayectoria} del cambio de estado se define especificando el estado inicial, la secuencia de estados intermedios que va tomando el sistema y el estado final.

Un \textit{proceso} \index{proceso} es el m\'etodo de operaci\'on mediante el cual se realiza un cambio de estado. La descripci\'on de un proceso consiste en establecer todo o parte de los siguientes: (1) la frontera; (2) cambio de estado, la trayectoria o los efectos producidos en el sistema durante cada etapa del proceso y (3) los efectos producidos en el medio externo durante cada etapa del proceso.

Supongamos que un sistema sometido a un cambio de estado regrese a su estado inicial. La trayectoria de esta transformaci\'on c\'{\i}clica se llama ciclo y el proceso mediante el cual se realiz\'o la transformaci\'on se llama \textit{proceso c\'{\i}clico}.

Una \textit{\gloss[word]{variabledeestado}} \index{estado!variable} es aquella que tiene un
valor definido cuando especifica el estado de un sistema.

En la siguiente parte se mostrar\'an los conceptos para poder responderse a preguntas tales como: ?`Qu\'e es un sistema?, ?`D\'onde existe una frontera?, ?`Cu\'al es el estado inicial?, ?`Cu\'al es el estado final?, ?`Cu\'al es la trayectoria de transformaci\'on?.


\subsection{Gases ideales} 
Los \gloss[word]{gasesideales}\index{gases ideales} son aquellos cuyo comportamiento no se
altera por fuerzas de cohesi\'on y vol\'umenes moleculares. Los \gloss[word]{gasesreales}
est\'an m\'as cercanos al comportamiento ideal mientras m\'as bajas sean las
presiones y m\'as altas las temperaturas.

\subsection{Ecuaciones de estado}

Las ecuaciones\index{estado!ecuaciones}  que representan las relaciones presi\'on
(\gloss[Word]{presion})- volumen (\gloss[Word]{volumen}) -
temperatura(\gloss[Word]{temperatura}) de los gases se conocen como \gloss[word]{ecuaciondeestado}, ya que describen completamente el estado de cualquier
sistema gaseoso. Las variables de estas ecuaciones se dividen en dos clases:
moles (\gloss[word]{moles}) y $V$ son variables extensivas (propiedades que dependen de la
cantidad de material) mientras que $P$ y $T$ son intensivas.

\begin{figure}[h]

\begin{center}
\begin{tabular}{||c|c|c|c|c|c||} \hline \hline
      &           &          &           &  & \\ \hline
      &           &          &           &  & \\ \hline
      &           &          &           &  & \\ \hline
      &           &          &           &  & \\ \hline
      &           &          &           &  &  \\ \hline \hline
\end{tabular}
\caption{Subdivisi\'on del sistema}
\label{fig:1}
\end{center}
\end{figure}


El valor de cualquier propiedad extensiva\index{estado!propiedad extensiva} se obtiene sumando
los va\-lores de la misma en todas partes del sistema. Supongamos que el sistema est\'a
subdividido en muchas partes peque\~nas, como en la Figura~\ref{fig:1}, entonces el volumen total del sistema se obtiene sumando los vol\'umenes de
cada una de las partes.

Las propiedades intensivas\index{estado!propiedad intensiva} no se obtienen mediante tal
proceso de su\-ma, sino que se miden de cualquier punto del sistema y cada una tiene un valor
uniforme en cualquier punto del sistema en equilibrio; por ejemplo, $T$ y $P$.

La temperatura ($T$) en todas las ecuaciones de estado es la \index{temperatura!absoluta} \gloss[word]{temperaturaabsoluta} es decir para el sistema m\'etrico decimal se encuentra en Kelvin\footnote{ Es importante identificar que la abreviatura $\kelvin$ de Kelvin es en may\'uscula mientras que de kilo (k) es en min\'uscula. As\'{\i} Kg representa Kelvin gramo y $\kilogram$ es kilogramo.} .

\begin{equation}
T[K]_{absoluta} = T[\degreecelsius] + 273.15
\end{equation}

\textit{Ley de \gloss[Word]{boyle}} \index{Boyle@\textbf{Boyle}!Ley de} Siempre que la masa
y la temperatura de una muestra de gas se mantienen constantes, el volumen del gas es
inversamente proporcional a su presi\'on absoluta.
\begin{equation}
P \propto \frac{1}{V} \quad{ \longrightarrow }\quad  P = \frac{k}{V}
\end{equation}
Tambi\'en podemos escribirla de la siguiente manera:
\begin{equation}
P_1 V_1 = P_2 V_2
\end{equation}
\begin{example}[Ley de Boyle]
 Un gas se comprime isotermicamente de $P_1= 1 \bbar$  y  $V= 10 \liter$  a una presi\'on de 2 \bbar. ?`Cu\'al es su volumen?\\[3pt]

$V_2 = \frac{P_1 V_1}{P_2}$\hskip.3in substituyendo\hskip.3in  $V_2 = \frac{1\bbar(10\liter)}{2 \bbar} = 5 \liter$
\end{example}

\textit{Ley de \gloss[Word]{charles}}\index{Charles@\textbf{Charles}} Siempre que la masa y la presi\'on de una muestra de gas se mantienen constantes, el volumen de la muestra es directamente proporcional a su temperatura absoluta.
\begin{equation}
V= k'T
\end{equation}
que tambi\'en se puede escribir de la siguiente forma:
\begin{equation}
\frac{V_1}{T_1} = \frac{V_2}{T_2}
\end{equation}

\begin{example} [Ley de Charles]
 A un gas ideal se le incrementa la temperatura a presi\'on constante de T = 15$\degreecelsius$ a 125 $\degreecelsius$ si su volumen
inicial es de 100 L. ?`Cu\'al es su volumen final?. \vskip0.1in
Para convertir la temperatura a absoluta se suman 273.15 $\kelvin$ as\'{\i} tenemos que 15$\degreecelsius$ son 288.15 y 125$\degreecelsius$ 398.15\kelvin
\vskip.2in
${V_2} = \frac{V_1 \times T_2}{T_1}$\hskip.25in sustituyendo\hskip.25in 
$V_2 = \frac{100\liter \times 398.15 \kelvin}{288.15 \kelvin} = 138.17 \liter$
\end{example}\vskip0.1in

\textit{Ley combinada de los gases} \index{ley combinada de los gases}. Empleando las ecuaciones anteriores se puede obtener la siguiente realci\'on entre el volumen, presi\'on y temperatura:
\begin{equation}
\frac{P_1V_1}{T_1} = \frac{P_2V_2}{T_2}
\end{equation}

 
 \subsection{Ley General de los gases}
 \index{ley!general de los gases}
 La ley general de los gases se puede aplicar para las condiciones de presi\'on y temperatura ambiente. La relaci\'on que guarda la temperatura, presi\'on, volumen y cantidad de masa se muestra en la siguiente ecuaci\'on:

\begin{equation}
PV = nRT\label{lgg}
\end{equation}

donde :

\qquad\begin{tabular}{c@{ -- }ll}
$P$ & presi\'on &[atm]\\
$V$ & volumen &[L]\\
$n$ & moles de gas &[\gram\mole]\\
\gloss[Word]{constR}& Constante de los gases&[\litre-atm/(\gram\mole \kelvin)]  \index{constante!ley general de los gases}\\
$T$ & temperatura absoluta&[\kelvin]\\
\end{tabular}

\begin{example}
?`Cu\'al es el volumen que ocupan 112 g de Nitr\'ogeno (N$_2$) a la presi\'on de la Ciudad de M\'exico (0.77 atm) y a una temperatura de $15 \degreecelsius$. (R = 0.082057 $\frac{\textrm{L-atm}}{\textrm{gmol} \kelvin}$)%


Convirtiendo los 15$\degreecelsius$ a absoluta tenemos 288.15\kelvin

Moles de N$_2$ = $\frac{112 \gram}{28 \frac{\gram}{\textrm{gmol}}} =$ 4 \gram\mole

$V = \frac{nRT}{P}$\hskip.10in substituyendo  \hskip.10in $V = \frac{4
\gram\mole \times 0.082057 \frac{\liter-\mathrm{atm}}{\gram\mole\, \kelvin} \times 288 \kelvin} {0.77 \mathrm{atm}}$\vskip0.1in

$V = 122.7$ \liter
\end{example}

y a partir de la \textbf{Ec.~\ref{lgg}} se puede despejar los t\'erminos para calcular la densidad de un gas mediante la siguiente ecuaci\'on :
\index{densidad!c\'alculo}
\begin{equation}
\rho = \frac{PM}{RT}
\label{ecdensidad}
\end{equation}

\begin{tabular}{llll}
Donde:\\
&\gloss[word]{densidad} & Densidad del gas & [$\frac{\kilo\gram}{\cubic\metre}$]\\
&$P$ & Presi\'on   &[atm]\\
&$M$ & Peso molecular del gas&[g/gmol]\\
&$R$ & Constante de los gases& [$\frac{\liter-\textrm{atm}}{\gram\mole\; \kelvin}$] \\
&$T$ & Temperatura  &[\kelvin] \\
\end{tabular}

\begin{example}[Ley general de los gases]%
 Calcular la densidad de un gas cuyo peso molecular es 29.98 g/gmol, a una presi\'on de
0.8 atm\'osferas y a una temperatura de 130\degreecelsius.\\
\begin{enumerate}
\item Calcular la temperatura en Kelvin.

 Si T[K] = T[\degreecelsius] $+ 273.15$ entonces  T[\kelvin] $= 130 +273.15 = 403.15$
\item Empleando la ecuaci\'on~\ref{ecdensidad} tenemos:

$\rho=\frac{PM}{RT}
=\frac{0.8(29.98)}{0.082057(303.15)} = 0.725\frac{\kilo\gram}{\cubic\metre} = 0.0452\frac{\mathrm{lb}}{\mathrm{ft}^3}$
\end{enumerate}
\end{example}

\subsection{Unidades empleadas en concentraciones atmosféricas}
 \index{unidades}
Los resultados de las mediciones de compuestos traza en el aire  realizadas para su estudio, evaluaci\'on y control se expresan en diferentes unidades, provenientes de distintos campos de estudio. En esta parte se presentan las unidades empleadas y sus diferentes factores de conversi\'on.

\vskip .15in
\begin{tabularx}{.9\linewidth}{XXXX}
\textbf{Longitud}&&\textbf{\'Area}\\
 1 pie (ft)& 0.3048 m& 1 pie (ft)$^2$ & 0.0929 m$^2$\\
 1 pie (ft)& 12 pulgada (in)\\
 1 pulgada (in)  & 2.54 cm\\
\textbf{Volumen}&&\textbf{Presi\'on}\\
1 pie$^3$ (ft$^3$)& 0.028317 m$^3$ &1 atm & 14.69 psi\\ 
1 pie$^3$ (ft$^3$)& 1728 in$^3$    &1 atm & 760 mmHg \\
1 pie$^3$ (ft$^3$)& 7.481 gal      &1 atm & 101,325 pa\\
\textbf{Masa}    &         &1 ``C.A. & 2.4908 pa\\
 1 lb& 7,000 gr            &1 ``C.A. & 0.03614 psi\\
 1 lb& 0.454 kg            &1 ``C.A. & 0.57818 onzas/in$^2$\\
\end{tabularx}

\begin{tabularx}{.9\linewidth}{XXXX}
\textbf{Constante }&\textbf{de los gases}&\textbf{Densidad}\\
0.082057 $\frac{\liter-\mathrm{atm}}{\gram\mole\, \kelvin}$&10.73
$\frac{\mathrm{lb}}{\mathrm{in}^2}\frac{\mathrm{pie}^3}{\mathrm{lbmol R}} $&1 $\frac{\mathrm{lb}}{\mathrm{pie}^3}$& 16.02
$\frac{\kilo\gram}{\metre^3}$\\\end{tabularx}
\index{constante!ley general de los gases}


Las conversiones entre diferentes unidades de concentraci\'on son las siguientes:

\begin{tabular}{lrl}
Multiplicar&por&para obtener\\\hline
1 ppm$_v$   &1 & \micro\mole\, de gas/\mole\, de gas\\
&1 $\times10^{-4}$ & porcentaje en volumen (\%$_v$)\\
&$\frac{\textrm{{\footnotesize Peso Molecular}}}{24.45}$&\milli\gram/\cubic\metre de aire\\
&&a (25\celsius\; y 1 atm )\\
&1 $\times10^{-6}$ &$\frac{\textrm{Presi\'on parcial de un
constituyente}}{\textrm{Presi\'on total de la mezcla}}$\\
ppb$_v$& 1 $\times10^{-3}$ & Partes por mill\'on (ppm)\\
1 \%$_v$&10 000&Partes por mill\'on (ppm)\\
mg/l &1 000&{\small miligramos sobre metro c\'ubico (mg/\cubic\metre)}\\
$\mu$g/m$^3$ & 1$\times 10^{-6}$ &miligramos sobre
litro (mg/\liter)\\\hline
\end{tabular}\index{concentraci\'on!unidades}
\vskip.1in
Para convertir concentraciones de gases y vapores de partes por mill\'on en volumen a miligramos por metro c\'ubico y viceversa a cualquier temperatura y presi\'on, se pueden emplear las siguientes ecuaciones:\index{concentraci\'on!conversiones}
\begin{eqnarray*}
\textrm{Concentraci\'on,}\; \textrm{ppm}&=&\frac{C_1\times24.45 \times
T}{{\textrm M}\times298\times P}\\
\textrm{Concentraci\'on,}\;\milli\gram/\cubic\metre&=& \frac{C_2\times {\textrm
M}\times298\times P}{24.45\times T}
\end{eqnarray*}
\begin{tabular}{llll}
Donde:\\
&$C_1$ & Concentraci\'on del gas & [\milli\gram/\cubic\metre]\\
&$C_2$ & Concentraci\'on del gas & [ppm$_v$]\\
&$T$ & Temperatura&[\kelvin]\\
&$M$ & Peso Molecular del gas&[\gram/\gram\mole]\\
&$P$ & Presi\'on&[atm]\\
\end{tabular} 
 
\subsection{Peso molecular promedio}
\index{peso molecular promedio}
El peso molecular promedio  ($\overline{\textrm{M}}$) , \'este se obtiene de sumar los valores obtenidos de multiplicar la fracci\'on mol ($y_i$) de cada compuesto con su respectivo peso molecular ($\textrm{M}_i$),
\begin{equation}
\overline{\textrm{M}}=\sum_{i=1}^ny_i\textrm{M}_i
\label{peso1}
\end{equation}

La fracci\'on mol ($y_i$)\index{fracci\'on mol} se puede obtener a partir del por ciento en volumen (\%$_v$) del gas en la mezcla de la siguiente forma:

\begin{equation}
y_i = \frac{\%_{vi} }{100} 
\label{fraccion}
\end{equation}

También se puede obtener la fracción molar de cada componente  como:

\begin{equation}
y_i = \frac{n_i}{\sum n_i}
\end{equation}

donde $n_i$ es el número de moles del gas $i$ en la mezcla.

Com\'unmente el an\'alisis de los gases se obtiene sin considerar la humedad del aire por lo cual se calcula el peso molecular promedio en base seca ($\overline{\textrm{M}}_s$) y a partir de conocer el porcentaje de humedad (\%H) en la mezcla se puede calcular el peso molecular base h\'umeda ($\overline{\textrm{M}}_w$). Mediante la siguiente relaci\'on:
\index{peso molecular promedio!base seca}
\index{peso molecular promedio!base h\'umeda}
\begin{equation}
\overline{\textrm{M}}_w= \frac{\%H}{100}\times \textrm{M}_{H_2O} +
\frac{100-\%H}{100} \times \overline{\textrm{M}}_s
\label{hum}
\end{equation}
Este cálculo es esencial en aplicaciones como la química atmosférica, la ingeniería de procesos y la termodinámica de gases, donde se requiere conocer el comportamiento global de una mezcla gaseosa en función de sus propiedades individuales. 

Además, el peso molecular promedio es un factor clave en la determinación de propiedades como la densidad de la mezcla, la velocidad de difusión de los componentes y la viscosidad del gas. En el estudio de la atmósfera, este valor permite comprender la variabilidad de la composición del aire en diferentes condiciones ambientales y su impacto en los procesos de transporte y reacción química en la troposfera y estratosfera.

\subsubsection{N\'umero de Avogadro (N$_A$)}
 \index{Avogadro!numero@n\'umero}
 
 El n\'umero de Avogadro representa el n\'umero de mol\'e\-culas o \'atomos en una mol de cualquier compuesto o elemento. Su valor es de 6.022$\times10^{23}$moleculas/mol.
 
\begin{example}[N\'umero de mol\'eculas de \ce{CO2}]%
 El \ce{CO2} ocupa alrededor de 354 ppm$_v$ de aire. ?`Cu\'antas mol\'eculas de \ce{CO2} hay en 1m$^3$ de aire a 1 atm y 0\celsius?\\
 A partir de $PV =nRT$  obtenemos $\frac{n}{V}=\frac{P}{RT}\cdot N_A=2.69\times10^{25} \frac{\textrm{molec}}{\cubic\metre}$\\
 Sabemos que$ \frac{\ce{Vol CO2}}{\ce{Vol Aire}}=\frac{\ce{Molec CO2}}{\ce{Molec Aire}}$ entonces:\\
\quad $\frac{354 \milli\liter}{\liter \cubic\metre}=\frac{3.54\times10^{-6}\cubic\metre}{1m^3}=\frac{\# molec \ce{CO2}}{2.69\times10^{25}}$\\
\quad n\'umero de moleculas de \ce{CO2}

$=3.54\times10^{-6}\cubic\metre\times2.69\times10^{25}\frac{\textrm{molec}}{\cubic\metre} =9.52\times10^{21}\textrm{molec}$\\
\end{example}
 
\subsubsection{Constante de Boltzman}
\index{Boltzman!constante}
Es la constante de los gases para N$_A$ mol\'eculas.

\[
\begin{array}{rcl}
k  &=&\frac{R}{N_A}  \\
k  &=&\frac{0.082057\frac{\litre-\mathrm{atm}}{\mole-\kelvin}}{6.022\times10^{23}}   \\    
k  &=&1.362\times10^{-25}\frac{\litre-\mathrm{atm}}{\mathrm{molecula}\: \kelvin}
\end{array}
\]

\begin{example}[Peso Molecular  Promedio]%
\index{peso molecular promedio!ejemplo}%
Calcular el peso molecular promedio del un flujo de gas de un proceso de calcinaci\'on\footnote{Proceso de calentar una sustancia a temperatura elevada, pero por debajo de su punto de fusi\'on.} que tiene la siguiente composici\'on: 15\% ox\'{\i}geno (\ce{O2}), 8\% bi\'oxido de carbono (\ce{CO2}),  69.55\% de nitr\'ogeno (N$_2$) y 7.45\% de agua (\ce{H2O}).\\
Aplicando las ecuaciones \ref{peso1} y \ref{fraccion} se genera la siguiente tabla donde se obtiene el peso molecular promedio $\overline{\textrm{M}}$\footnote{\%$_v$ --- es el por ciento en volumen del compuesto en el gas\\
$y_i$ --- es la fracci\'on volumen o mol del compuesto $i$ en la mezcla gaseosa\\
 M$_i$ --- es el peso molecular del compuesto $i$ \\
 $y_i\times M_i$ --- es la fracci\'on de peso molecular del compuesto $i$ en la mezcla gaseosa.} 
\begin{footnotesize}

\begin{tabular}{c r@{.}l r l r@{.}l}
Compuesto& \multicolumn{2}{c}{\%$v$ } &$y$ & M$_i$&\multicolumn{2}{c}{$y\times M$}\\\hline 
\ce{O2}    &   15&0  &0.150& 32   & 4&800\\
\ce{CO2}   &    8&0  &0.080& 44   & 3&520\\
\ce{N2}    &   69&55 &0.695& 28   &19&474\\
\ce{H2O}   &    7&45 &0.075& 18   & 1&341\\ \hline
 $\overline{\textrm{M}}$ &\multicolumn{2}{c}{}&&&\textbf{29}&\textbf{135}\\
\end{tabular}
\end{footnotesize}

Sumando los productos $y_i\times M_i$ se obtiene que el peso molecular promedio es 29.1 \gram/\gram\mole
\end{example}

\begin{example}[Peso Molecular de Gas de Combusti\'on]%
Se tiene un gas de combusti\'on cuya composici\'on es: 12\% ox\'{\i}geno (\ce{O2}) y 22\% bi\'oxido de  carbono (\ce{O2}). Mediante otro an\'alisis se determin\'o que la humedad del gas es del 12\%. Calcular el peso molecular.
\begin{enumerate}
\item El nitr\'ogeno (\ce{N2}) en la mezcla gaseosa se obtiene por diferencia, as\'{\i}
tenemos que:
 \%\ce{N2}$ = 100 - $\%\ce{O2}$-$ \%\ce{CO2}$\;$ entonces  \%\ce{N2}$ = 100 - 12 - 22 = 66$.
\item Se calcula el peso molecular promedio base seca ($\overline{\textrm{M}}_s$)

\begin{footnotesize}
\begin{tabular}{c r@{.}l r l r@{.}l}
Compuesto& \multicolumn{2}{c}{\% Vol} & $y$ &M&\multicolumn{2}{c}{$y\times M$}\\\hline 
\ce{O2}    &   12&0 &0.12& 32   &  3&84\\
\ce{CO2}   &   22&0 &0.22& 44   &  9&68\\
\ce{N2}    &   66&0 &0.66& 28   & 18&48\\ \hline
         &\multicolumn{2}{c}{}&&& 32&00 \\
\end{tabular}
\end{footnotesize}
\item Se calcula el peso molecular base h\'umeda ($\overline{\textrm{M}}_w$) mediante 
~\ref{hum}.
$\overline{\textrm{M}}_w=\frac{12}{100}\times18 + \frac{100-12}{100}\times32$
$\overline{\textrm{M}}_w= 30.32$ g/gmol
\end{enumerate}
\end{example}

 \subsection{Gases no ideales}
Sin embargo en la realidad los gases no se comportan ``idealmente'' y as\'{\i} esta ecuaci\'on sirve en condiciones de presi\'on baja. La \textit{temperatura o punto de Boyle} es aquella temperatura para la cual un gas ideal se comporta en forma ideal en un amplio intervalo de presiones. \index{Boyle@\textbf{Boyle}!punto de}

Existen otras ecuaciones que se utilizan para calcular el com\-por\-tamien\-to de los gases reales como lo son:

Uso del \gloss[word]{factordecompresibilidad} \gloss[word]{fcomp}:
\index{compresibilidad!factor de}
\begin{equation}
PV = zn RT
\end{equation}

La ecuaci\'on de Estado de \textbf{Van der Waals} \index{Van der Waals@\textbf{Van der Waals}}
\begin{equation}
(P+ \frac{n^2a}{V^2})(V-nb)= nRT
\end{equation}
Esta ecuaci\'on difiere de la de los gases ideales, en que considera tanto   el volumen ocupado por las propias mol\'eculas, como de las fuerzas
atractivas existentes entre las mismas. Algunos valores de \gloss[word]{awaals} y \gloss[word]{bwaals} se muestran en el \textbf{Cuadro~\ref{tab:1}}.

\begin{table}[ht]
\begin{minipage}{\linewidth}
\caption[Constantes de van der Waals]{{\small Constantes de Van der Waals para varios gases.}\footnote{Maron, S. H., \& Prutton, C. F. (1990). \textit{Fundamentos de fisicoquímica}. Editorial Limusa.}}
\begin{center}
{\small \begin{tabular}{lccc}\hline
Gas&F\'ormula&a                                         &b  \\
      &                &atm-\liter$^2$ \mole$^{-2}$ & \liter \mole$^{-1}$ \\\hline
Amon\'{\i}aco       & \ce{NH3}    & 4.17  & 0.0371 \\
Arg\'on             & Ar        & 1.35  & 0.0322 \\
Bi\'oxido de Carbono& \ce{CO2}    & 3.59  & 0.0427 \\
Etano               & \ce{C2H6}& 5.49  & 0.0638 \\
Hidr\'ogeno         & \ce{H2}     & 0.244 & 0.0266\\
Metano              & \ce{CH4}    & 2.25  & 0.0428 \\
Nitr\'ogeno         & \ce{N2}     & 1.39  & 0.0391 \\
Ox\'{\i}geno        & \ce{O2}    & 1.36  & 0.0318 \\
Agua                & \ce{H2O}    & 5.46  & 0.0305 \\ \hline
\end{tabular}}
\end{center}
\label{tab:1}
\end{minipage}
\end{table}

La ecuaci\'on de estado de \textbf{Karmerligh Onnes} \index{Karmerligh Onnes@\textbf{Karmerligh}} Esta ecuaci\'on expresa el producto $PV$ como una serie de potencias de presi\'on, a cualquier temperatura dada, esto es:
\begin{equation}
PV_m =A +BP + CP^2 + DP^3 + \ldots
\end{equation}
donde $P$ es la presi\'on generalmente en atm\'osferas, $V_m$ el volumen
molar en litro. Los coeficientes $A$, $B$, $C$, etc., son conocidos como el primero, segundo, etc., \textit{coeficiente virial}


\begin{exercises}
\exer Si a un gas ideal a $T$ = 25{\degreecelsius}, $P$ = 1 atm y $V$ = 100\liter, se calienta a 200{\degreecelsius} y se encuentra a una presi\'on de 1.1 atm . ?`Cu\'al es ahora su volumen final?. 

\exer ?`Qu\'e volumen ocupa un mol de gas ideal a una presi\'on de una atm\'osfera y \subexer Temperatura de 0\degreecelsius 
\subexer Temperatura de 25{\degreecelsius}.

\exer Dos gramos de ox\'{\i}geno se encuentran encerrados en un recipiente de 2\liter a una presi\'on de 121 \kilo\pascal?`Cu\'al es la temperatura del gas en $\degreecelsius$? 

\exer Cinco gramos de etano (\ce{CH3CH3}) se encuentran dentro de un bulbo de un litro de capacidad. El bulbo es tan d\'ebil que se romper\'a si la presi\'on sobrepasa las 10 atm\'osferas. ?`A qu\'e temperatura alcanzar\'a la presi\'on del gas el valor de rompimiento?

\exer Trazar en papel milim\'etrico los siguientes datos:
 
{\centering
\begin{tabular}{lcrrrrr} \hline
\textbf{Volumen}& \milli\liter &  4 & 5& 6& 7& 8 \\ 
\textbf{ Presi\'on (abs)}ml & atm&   1.140& 0.912 & 0.759 & 0.651& 0.570 \\ \hline
\end{tabular}
\vskip6pt}
Sobre el eje de las $x$ va la presi\'on y en el eje de las  $y$ el volumen. A partir de la gr\'afica responda las  siguientes preguntas:
\subexer ?`C\'omo se comporta el volumen al incrementarse la presi\'on?
\subexer Complete la siguiente oraci\'on: a mayor presi\'on \hrulefill volumen.
\subexer Calcule la pendiente de la l\'{\i}nea trazada.
\exer Sobre el eje de las $x$ va la temperatura y en el eje de las $y$ el volumen. A partir de la gr\'afica responda las
\begin{center}
\begin{tabular}{cc} \hline
\textbf{Volumen} &\textbf{ Temperatura} \\ 
ml & \kelvin \\ \hline
31.60 & 293\\
32.39& 300 \\
33.15 & 307 \\
34.01 & 315 \\
34.55 & 320 \\ \hline
\end{tabular}
\end{center}

siguientes preguntas:

\subexer ?`C\'omo se comporta  el volumen al incrementarse la
temperatura?
\subexer Complete la siguiente oraci\'on: a mayor temperatura \hrulefill
\hskip 1mm volumen.
\subexer Calcule la pendiente de la l\'{\i}nea trazada.


\exer Conversi\'on de unidades

\subexer  ?`Cu\'anto equivale en pascales (Pa) 585 mmHg?
\subexer ?`Cu\'anto es 23.9 calor\'{\i}as en eV?
\subexer Una persona promedio debe disipar energ\'{\i}a a un
r\'egimen promedio de 110 W para permanecer viva.  ?`Cu\'antas kilocalor\'{\i}as por d\'{\i}a debe consumir
esta persona para continuar viva? (Tip: d\'{\i}a = 24h) 
\subexer  ?`Cu\'anto sería la concentración de 0.090 ppm de ozono en \milli\gram / \cubic\metre ?
\exer Peso molecular promedio y densidad
\subexer ?`Cu\'al sería el peso molecular promedio del aire atmosférico si se considera que tiene 21\% de oxígeno y 79\% de nitrógeno? 
\subexer ?`Cu\'al sería la densidad del aire a una temperatura de 20\celsius\, y 1 atm de presión ? 
\end{exercises}

\section{Primera ley de la termodin\'amica}
\index{termodinamica@termodin\'amica!primera ley de la }
La primera ley de la termodin\'amica establece la conservaci\'on de la
energ\'{\i}a, es decir, la energ\'{\i}a no se crea, ni se destruye. En otras
palabras, esta ley se formula diciendo que para una cantidad dada de
una forma de e\-nerg\'{\i}a que desaparece otra forma de la misma
aparecer\'a en una cantidad igual a la cantidad desaparecida.
Si cierta cantidad de calor \gloss[word]{qcalor} es agregada a un sistema, esta cantidad dar\'a origen a un incremento de la energ\'{\i}a interna del sistema
y tambi\'en efectuar\'a cierto trabajo externo como consecuencia de dicha absorci\'on calor\'{\i}fica. Si designamos por \gloss[word]{deltae} al incremento de
energ\'{\i}a del sistema y \gloss[word]{wtrab} al trabajo realizado por el sistema sobre el contorno, entonces por la primera ley tendremos:
\begin{equation}
\Delta E + w = Q
\end{equation}
y
\begin{equation}
\Delta E = Q - w
\end{equation}
\paragraph{Trabajo}\index{trabajo} En termodin\'amica \gloss[word]{trabajo} se define como la cantidad de energ\'{\i}a que fluye a trav\'es de la frontera de un sistema durante un cambio de estado y que se puede usar por completo para elevar un cuerpo en el medio exterior. Es importante hacer notar que:
\begin{enumerate}
\item El trabajo s\'olo aparece en la frontera de un sistema.
\item El trabajo s\'olo aparece \textit{durante} un cambio de estado.
\item El trabajo se manifiesta por su efecto en el \textit{ medio exterior}.
\item La cantidad de trabajo es igual $mgh$, donde \gloss[word]{masa} es la masa elevada,
\gloss[word]{ggravedad} la aceleraci\'on debida a la gravedad, \gloss[word]{haltura} es la
altura a que se ha elevado el cuerpo.
\item El trabajo es una cantidad algebraica; es una transferencia de energ\'{\i}a que no se debe a una diferencia de temperatura. Es positivo si ha elevado un cuerpo en el exterior, en cuyo caso se dice que el trabajo se ha producido en
el medio exterior o que ha fluido \textit{hacia} el medio exterior; es
negativo si ha descendido un cuerpo ($h$ es --) en cuyo caso decimos que el
trabajo se ha destruido en el medio exterior o que ha fluido
\textit{desde} el medio exterior.

\end{enumerate}
\paragraph{Calor} El proceso mediante el cual se logra el equilibrio de dos
sistemas es mediante el flujo de calor $Q$ que fluye del sistema de
temperatura m\'as elevada al sistema de temperatura inferior.

En termodin\'amica se define \gloss[word]{calor} \index{calor}como una cantidad de energ\'{\i}a que  fluye a trav\'es de la frontera de un sistema durante un cambio de estado, en virtud de una diferencia de temperatura entre el sistema y su medio exterior, y que fluye de un punto de mayor a uno de menor temperatura.

Deben se\~nalarse diversos aspectos de esta definici\'on:
\begin{enumerate}
\item El calor s\'olo aparece en la frontera del sistema
\item El calor s\'olo aparece \textit{durante} el cambio de estado.
\item El calor se manifiesta por un efecto en el \textit{medio exterior}.
\item En un calor\'{\i}metro, la cantidad de calor se manifiesta en el incremento de la temperatura de la masa de agua  partiendo de una temperatura espec\'{\i}fica bajo un volumen constante. Existen calor\'{\i}metros a presi\'on constante donde intervienen reacciones qu\'{\i}micas donde puede ocurrir absorci\'on o desprendimiento de calor y entonces se refiere a un cambio en la temperatura.  
\item El calor es una cantidad algebraica; es positivo si una masa de agua en el medio exterior se enfr\'{\i}a, en cuyo caso decimos que est\'a fluyendo calor \textit{desde} el medio exterior; es negativo  si se calienta una masa de agua en el medio exterior y se dice entonces que fluye calor \textit{hacia} el medio exterior.
\end{enumerate}

\begin{table}[ht]
\caption{Diferencias entre calor y temperatura}
\label{tab:2}
\begin{center}
{\small \begin{tabularx}{.9\linewidth}{XX} \hline
   \hskip .7in \textbf{Calor}& \hskip .4in   
\textbf{Temperatura}\\  \hline
\textit{Energ\'{\i}a en transferencia} que  percibe el tacto al tocar un objeto 
 & \textit{\'Indice} que se mide con un term\'ometro\\ 
 
Energ\'{\i}a que entra o sale de un sistema y se relaciona a un \textit{movimiento ca\'otico}
de las mol\'eculas&
 Propiedad termodin\'amica  proporcional a la \textit{energ\'{\i}a} \textit{cin\'etica
promedio} de las mol\'eculas\\ 
\textit{Depende de la masa}: el calor suministrado a 100 g de agua para calentarlo a
10$^\circ$C es diferente (la mitad) del necesario para calentar 200 g de agua. &\textit{No depende de
la masa}: la temperatura final de los 100 y 200 g de agua es la misma, 20$^\circ$C. \\ 
Se mide en un \textit{\gloss[word]{calorimetro}} \index{calor\'{\i}metro}&Se mide con un
\textit{term\'ometro} \index{term\'ometro}\\
Sus unidades son \textit{Joule} y \textit{calor\'{\i}as.} &
 Sus unidades son \textit{Kelvin} y grados \textit{Celsius}\\ \hline
\end{tabularx}}
\end{center}
\end{table}

\subsection{Energ\'{\i}a interna y entalp\'{\i}a}

Las mol\'eculas de un sistema poseen movimientos de rotaci\'on, vibraci\'on,
y translaci\'on que dan lugar a la energ\'{\i}a cin\'etica correspondiente.
La suma de las energ\'{\i}as cin\'eticas moleculares y la energ\'{\i}a
potencial molecular se expresa como una energ\'{\i}a propia del cuerpo o energ\'{\i}a interna:
\begin{equation}
E_\textrm{interna}=E_\textrm{estructura}+E_\textrm{rotaci\'on}+E_\textrm{vibraci\'on}+E_\textrm{traslaci\'on}
\end{equation}

Le energ\'{\i}a interna es la energ\'{\i}a total de las mol\'eculas que componen el sistema.\index{energ\'{\i}a!interna}

 La propiedad que contabiliza los cambios t\'ermicos a presi\'on constante en un sistema se
llama \textit{\gloss[word]{entalpia}} (\gloss[word]{H}) \index{entalp\'{\i}a} y se define
como:
\begin{equation}
H \equiv  E + PV
\end{equation}
donde $P$ y $V$ son la presi\'on y el volumen del sistema respectivamente. Los
cambios en el calor y trabajo de un sistema se les conoce como cambios t\'er\-mi\-cos.


El calor no se contabiliza internamente como tal sino como un cambio en la energ\'{\i}a interna $\Delta E$. De igual manera conviene
contabilizar el trabajo representado por su efecto $PV$ donde $P$ es
la presi\'on externa y  \gloss[word]{deltav} es el cambio de volumen.

Algunos de los procesos donde existen cambios t\'ermicos son:

\begin{center}
\begin{tabular}{ll}
Entalp\'{\i}a de reacci\'on (\gloss[word]{deltahr}).  & Entalp\'{\i}a de evaporaci\'on (\gloss[word]{deltahev}).\\
Entalp\'{\i}a de disoluci\'on (\gloss[word]{deltahd}). &Entalp\'{\i}a de fusi\'on (\gloss[word]{deltahf}). \\
Entalp\'{\i}a de combusti\'on (\gloss[word]{deltahc}). &  Entalp\'{\i}a de enlace (\gloss[word]{deltahen}).\\
Entalp\'{\i}a de formaci\'on (\gloss[word]{deltahfr}). &
\end{tabular}
\end{center}


\paragraph{Entalp\'{\i}a de reacci\'on $(\Delta H_r)$}\index{entalp\'{\i}a!reacci\'on}
Calor que se absorbe o libera cuando los reactivos se convierten en
productos.
\paragraph{Entalp\'{\i}a de disoluci\'on $(\Delta H_d)$}\index{entalp\'{\i}a!disoluci\'on}
Calor que se absorbe o libera cuando se di\-suelve una mol de soluto
en una cantidad fija de disolvente a una temperatura determinada.
\paragraph{Entalp\'{\i}a de combusti\'on $(\Delta H_c)$}\index{entalp\'{\i}a!combusti\'on}
Calor que se absorbe o libera cuando se quema 1 mol de combustible.
\paragraph{Entalp\'{\i}a de formaci\'on $(\Delta H_f)$}\index{entalp\'{\i}a!formaci\'on}
Calor que se absorbe o libera cuando se forma 1 mol de una sustancia a
partir de sus elementos.
\subparagraph{Unidades de energ\'{\i}a}
\index{energ\'{\i}a!unidades}

1 cal = 4.185 \joule\\
{\small 1 \joule = $10^7$erg = 0.239 cal = 2.78$\times10^7$kWh =
6.242$\times10^{18}$ e\volt = 9.8$\times10^{-3}$\liter-atm}

\begin{example}
Si la grasa animal contiene 40,000 $\frac{\joule}{\gram}$ de energ\'{\i}a  ?`Cu\'anto corresponde en kcal/g?

\hskip 1.in 40,000 $\frac{\joule}{\gram}$ x  $\frac{0.239 cal}{\joule}$ x $\frac{1 \textrm{ kcal}}{1,000
\textrm{ cal}}$ = 9.56 $\frac{\textrm{kcal}}{\gram}$
\end{example}

\begin{example}
 El calor de combusti\'on del carbono es de $-94$ {kcal/\mole} entonces ?`a cu\'an\-tos  {\kilo\joule/\mole} equivale?

\hskip 1in -94$\frac{\textrm{kcal}}{\mole}\times\frac{4,184 \kilo\joule}{1  \textrm{ kcal}} =$ -393
$\frac{\kilo\joule}{\mole}$
\end{example}
\begin{example}
Se tiene un sistema cuya entalp\'{\i}a inicial ($H_1$) es de 120 cal, despu\'es de una
reacci\'on entre los componentes del sistema se tiene una $H_2=$1,230 cal, ?`Cu\'al es la
diferencia de energ\'{\i}a del sistema entre los dos estados? 

\begin{tabular}{lll}
$H_1=$ 120 cal & $\Delta H = H_2 -H_1$ &$\Delta H = 1,230 -120$ \\
$H_2=$ 1,230 cal && $\Delta H = 1,110$ cal
\end{tabular}
\end{example}

\noindent\textbf{\gloss[Word]{capacidadcalorif}} (\gloss[word]{cp}) Es la cantidad de calor que se requiere para elevar un grado Celsius la temperatura de un gramo de la sustancia.
\index{capacidad@capacidad calor\'{\i}fica|textbf}
\begin{equation}
\textrm{Q} =\textrm{m} \textrm{C}_p \Delta \textrm{T}
\end{equation}
 donde
Q es el calor  (kcal) suministrado a una masa m(\gram) con capacidad calor\'{\i}fica de C$_p$ ($\frac{\textrm{kcal}}{^\circ \textrm{C}\,\gram}$) el cual hace que se incremente la temperatura en $\Delta$T.

\subsection{Reacciones exot\'ermicas y endot\'ermicas}

Dependiendo de los reactivos que participan en una reacci\'on se rompen enlaces y se forman nuevos, durante este proceso existe un intercambio de energ\'{\i}a, tanto al inicio de la reacci\'on como durante la misma, as\'{\i} tenemos que una reacci\'on \gloss[word]{exotermico} es aquella que al efectuarse libera calor ($\Delta H -$)
\index{reaccion@reacci\'on!exot\'ermica} y una reacci\'on \gloss[word]{endotermico} es aquella que para efectuarse necesita calor ($\Delta H +$) \index{reaccion@reacci\'on!endot\'ermica}

\begin{example}[Exot\'ermicas ] 
Combusti\'on, combinaci\'on de un \'acido con una base, disoluci\'on de \'acidos o bases, reacci\'on del \'oxido de calcio (\ce{CaO}) con agua para producir el hidr\'oxido de calcio (\ce{Ca(OH)2}).
\end{example}
\begin{example}[Endot\'ermicas]
 Disoluci\'on de sales, descomposici\'on de \'oxidos. Calcinaci\'on del carbonato de calcio (\ce{CaCO3}) para producir \'oxido de calcio (\ce{CaO}).
\end{example}

\begin{table}[htb!]
\begin{minipage}{\linewidth}
\caption{Algunas reacciones exot\'ermicas y endot\'ermicas a 25\degreecelsius}
\label{exo-endo}
\begin{center}
{ \begin{tabular}{lclll}\hline
Reactivos\footnote{Fuente: Castellan, G. W. (1986). \textit{Fisicoquímica}. Fondo Educativo Interamericano.}
&&Productos&Energ\'{\i}a&Tipo\\ \hline
\ce{I2(g) +H2(g)}&\ce{->} & \ce{2HI} & $\Delta H = +6.2 $\kilo cal& endot\'ermica \\
\ce{1/2N2(g) + 1/2O2(g)} &\ce{->} &  \ce{NO(g)} & $\Delta H = +21.6 $\kilo cal & endot\'ermica \\
\ce{1/2H2(g) + 1/2Cl2(g)}&\ce{->} & \ce{HCl((g)}  & $\Delta H = -22.06 $\kilo cal& exot\'ermica \\
\ce{Na(g) +1/2Cl(g)}        &\ce{->} & \ce{NaCl(g)}  & $\Delta H = -98.23 $\kilo cal& exot\'ermica \\
\ce{H2(g) + 1/2O2(g)}    &\ce{->} &  \ce{H2O(g)} & $\Delta H = -57.8 $\kilo cal& exot\'ermica \\\hline
\end{tabular} }
\end{center}
\end{minipage}
\end{table}

En el \textbf{Cuadro~\ref{exo-endo}} se muestran ejemplos de reacciones tanto exot\'ermicas como endot\'ermicas, como se puede observar las reacciones exot\'ermicas tienen valores negativos de entalp\'{\i}a de reacci\'on  mientras las endot\'ermicas positivos.
\clearpage

\subsection{Entalp\'{\i}as de enlace} \index{entalp\'{\i}a!enlace}
La entalp\'{\i}a de enlace es el cambio de entalp\'{\i}a necesario para romper un enlace de un mol de mol\'eculas gaseosas. Mediante \'esta se pueden  evaluar los calores de reacci\'on de un proceso para el cual no existen datos t\'ermicos disponibles, es aplicable a las reacciones gaseosas entre sustancias que tiene s\'olo enlaces covalentes, y se basa en los siguientes supuestos:

a) Todas las entalp\'{\i}as de enlace de un tipo particular, como el metano \ce{C-H} son id\'enticas, y

b) Las entalp\'{\i}as de enlace son independientes de los compuestos en que aparecen.

Aunque ninguna suposici\'on es v\'alida estrictamente, sin embargo el m\'etodo ofrece un procedimiento simple y bastante satisfactorio para determinar las entalp\'{\i}as de varias reacciones. Un conjunto promedio de entalp\'{\i}as de enlace (\gloss[word]{deltahe}) y sus mejores valores como  los que se muestran en la \textbf{Cuadro~\ref{tab:3}}.

\begin{table}
\begin{minipage}{\linewidth}
\caption[Entalp\'{\i}as de enlace]{Valores emp\'{\i}ricos de las entalp\'{\i}as de enlace a  25$\degreecelsius$ \footnote{ Maron, S. H., \& Prutton, C. F. (1990). }}
\begin{center}
{\small \begin{tabular}{lccc} \hline
\textbf{Enlace}&\textbf{$\Delta$H } &\textbf{Enlace}&\textbf{$\Delta$H}\\ 
&kcal/mol & &kcal/mol\\ \hline
 \ce{H-H}  & 104  &  \ce{C-Cl}    &  79 \\ 
 \ce{H-F}   & 135  & \ce{C-Br}    &  66 \\ 
 \ce{H-Cl}   & 103  & \ce{C-S}    &  62 \\ 
 \ce{H-Br}   &  88  &  \ce{C=S}     & 114 \\
\ce{ O-O} &  33  & \ce{C-N}     &  70 \\
 \ce{O=O}  & 118  & \ce{C=N}     & 147 \\
 \ce{O-H}  & 111  & \ce{C\bond{#}N}& 210 \\
 \ce{C-H}  &  99  & \ce{N-N}     &  38 \\
 \ce{C-O}  &  84  & \ce{N=N}     & 100 \\
 \ce{C=O}  & 170  & \ce{N\bond{#}N}& 226 \\
 \ce{C-C}  &  83  & \ce{N-H}     & 93  \\
 \ce{C=C}   & 147 & \ce{F-F}     & 37 \\
 \ce{C\bond{#}C}&194& \ce{Cl-Cl}   & 58 \\
 \ce{C-F}   & 105  & \ce{Br-Br}  & 46  \\
&&\ce{C (s, grafito) =  C(g)}&172 \\
 \hline 
\end{tabular}}
\label{tab:3}
\end{center}
\end{minipage}
\end{table}

Al usar estas entalp\'{\i}as un signo m\'as se adscribe a la entalp\'{\i}a de enlace roto, porque para ello precis\'o absorci\'on de calor, y el signo menos se utiliza cuando se produce un enlace y hay desprendimiento calor\'{\i}fico.
\pagebreak 
\begin{example}
Supongamos primero, que se busca el cambio ent\'alpico a 25$^\circ$
C de la reacci\'on
\begin{equation}
\ce{H2C=CH2 (g) + H2 (g) -> CH3-CH3 (g)}
\end{equation}
En este caso cuatro enlaces \ce{C-H} en el \ce{C2H4} est\'an inafectados y pueden despreciarse. Sin embargo, un doble enlace se rompe en \ce{C2H4}  y uno \ce{H-H }en \ce{H2}. A su vez, en el \ce{C2H6} se produce un enlace \ce{C-C}  y dos \ce{C-H}; entonces para $\Delta H$ tenemos:

\begin{equation}
\label{eq:1}
\Delta H_{25\degreecelsius} = - (\Delta H_{\ce{C-C}} + 2 \Delta H_{\ce{C-H}}) + (\Delta H_{\ce{C=C}} + \Delta H_{\ce{H-H}})
\end{equation}

Al sustituir los valores en la ecuaci\'on~\ref{eq:1} las entalp\'{\i}as de enlace de la \textbf{Cuadro~\ref{tab:1}}, tenemos

\hskip 1in   $\Delta H_{25\degreecelsius} = -(83 \,\textrm{kcal} + 198 \,\textrm{kcal}) + (147 \,\textrm{kcal} + 104 \,\textrm{kcal})$

\hskip 1.59in $= -30 \,\textrm{kcal}$
\end{example}
\begin{example}
 Para la reacci\'on
\begin{equation}
\ce{2H2   +  O2 ->  H2O(g)}
\end{equation}

Para esta reacci\'on dos enlaces \ce{H-H} del \ce{H2} se rompen y un enlace \ce{O=O} del \ce{O2}. A s\'{\i} mismo se producen cuatro enlaces \ce{O-H}; entonces tenemos:
\begin{equation}
\label{eq:2}
\Delta H_{25\degreecelsius} = - 4\Delta H_{\ce{O-H}} 
+ (\Delta H_{\ce{O=O}} + 2\Delta H_{\ce{H-H}})
\end{equation}
Al sustituir los valores en la ecuaci\'on~\ref{eq:2} las
entalp\'{\i}as de enlace de la \textbf{Cuadro~\ref{tab:1}}, tenemos:

\hskip 1in   $\Delta H_{25\degreecelsius} = -4(111 \,\textrm{kcal}) + (118 \,\textrm{kcal} + 208 \,\textrm{kcal})$

\hskip 1.59in $= -118  \,\textrm{kcal}$
\end{example}
\begin{example}
En la reacci\'on de \gloss[word]{halogenacion} \index{halogenaci\'on}
tenemos
\begin{equation}
\ce{2Br2 + CH=CH ->  CHBr2-CHBr2(g)}
\end{equation}

Para esta reacci\'on dos enlaces \ce{Br-Br} del \ce{Br2} se rompen y un enlace \ce{C\bond{#}C} del \ce{C2H2}. A s\'{\i} mismo se producen cuatro enlaces \ce{C-Br}; entonces tenemos:
\begin{equation}
\label{eq:3}
\Delta H_{25\degreecelsius} = - 4\Delta H_{\ce{C-Br}} 
+ (\Delta H_{\ce{C\bond{#}C} } + 2\Delta H_{\ce{Br-Br}})
\end{equation}
Sustituyendo los valores de las entalp\'{\i}as de enlace de la \textbf{Cuadro~\ref{tab:1}} en la e\-cua\-ci\'on~\ref{eq:3} , tenemos:

\hskip 1in   $\Delta H_{25\degreecelsius} = -4(66 \,\textrm{kcal}) + (194 \,\textrm{kcal} + 92 \,\textrm{kcal})$

\hskip 1.59in $= 22 \,\textrm{kcal}$
\end{example}

\subsection{Termoqu\'{\i}mica. Ley de Hess}

La termoqu\'{\i}mica es una parte de la qu\'{\i}mica que estudia la relaci\'on del calor con las reacciones qu\'{\i}micas. Su objetivo es
la determinaci\'on de las cantidades de energ\'{\i}a calor\'{\i}fica cedida o captada en los distintos procesos y el desarrollo de
m\'etodos de c\'alculo de dichos ajustes sin recurrir a la experimentaci\'on.

\paragraph{Ley de Hess} \index{Hess@\textbf{Hess}!Ley de}El calor desprendido o absorbido en una reacci\'on dada debe ser independiente de la manera particular como se verifica. Con otras palabras si una reacci\'on procede en varias etapas, el calor de reacci\'on total ser\'a la suma algebraica de los calores de las distintas etapas, y a su vez esta suma es id\'entica a la que tendr\'{\i}a lugar por absorci\'on o desprendimiento en una reacci\'on que procediera en una sola etapa.

Lo anterior se ilustra con los siguientes \textit{ejemplos}:

\begin{example}
Sup\'ongase que comparamos dos m\'etodos diferentes para la s\'{\i}ntesis del cloruro de sodio a partir de sodio y cloro.

{\footnotesize \begin{tabular}{lrclr}
\textit{M\'etodo 1:}&&&\\
&
$\frac{1}{2}$H$_2${\footnotesize(g)}+$\frac{1}{2}$Cl$_2${\footnotesize
(g)}&
$\longrightarrow$&
HCl{\footnotesize (g)} &$\Delta H= -22.06$ \,\textrm{kcal} \\
&Na{\footnotesize (s)} + HCl{\footnotesize (g)}&
$\longrightarrow$&
NaCl{\footnotesize (s)} + $\frac{1}{2}$H$_2${\footnotesize (g)}& 
$\Delta H= -76.17$ kcal\\ \hline
{\footnotesize Cambio Neto:}& Na{\footnotesize (s)} + $\frac{1}{2}$Cl$_2${\footnotesize (g)}&
$\longrightarrow$&
NaCl{\footnotesize (s)}& $\Delta H= -98.23$ kcal \\
\end{tabular}}
\vskip .15in
{\footnotesize \begin{tabular}{lrclr}
\textit{M\'etodo 2:}&&&\\
&Na{\footnotesize (s)} + H$_2$O{\footnotesize (l)}&$\longrightarrow$ &
 NaOH{\footnotesize (s) + $\frac{1}{2}$H$_2${\footnotesize
(g)}}& $\Delta H= -33.67$ kcal \\ & $\frac{1}{2}$H$_2${\footnotesize
(g)}+$\frac{1}{2}$Cl$_2${\footnotesize (g)}&
$\longrightarrow$&
HCl{\footnotesize (g)} & $\Delta  H= -22.06$ kcal \\
& HCl{\footnotesize (g)  + NaOH{\footnotesize (s)} }&
$\longrightarrow$&
 NaCl{\footnotesize (s) + H$_2$O{\footnotesize (l)}} &
$\Delta  H= -42.50$ kcal \\ \hline 
{\footnotesize Cambio Neto:}&
Na{\footnotesize (s)} + $\frac{1}{2}$Cl$_2${\footnotesize (g)}&
$\longrightarrow$&
NaCl{\footnotesize (s)}&  $\Delta  H= -98.23$ kcal \\
\end{tabular}}
\vskip .15in
El cambio qu\'{\i}mico neto se obtiene sumando todas las reacciones de la secuencia. La variaci\'on neta de entalp\'{\i}a \index{entalp\'{\i}a!Hess} se obtiene sumando todas las variaciones de entalp\'{\i}a en la secuencia. La variaci\'on en la de entalp\'{\i}a debe ser la misma para cualquier secuencia en que resulte el mismo cambio neto.\vskip0.2in
\end{example}

\begin{example}
 Se desea encontrar la $\Delta$H de la reacci\'on
 
%\hskip1in$2$C{\footnotesize (s)}$ + 2$H{\footnotesize (g)}$+$O$_2${\footnotesize(g)}$\longrightarrow$CH$_3$ COOH{\footnotesize (l)}  \hskip.12in $\Delta  H = ?$
\hskip0.25in\ce{2C (s) + 2H2 (g) +O2 (g) -> CH3COOH (l)} \hskip.12in $\Delta  H = ?$

que no se puede determinar directamente. Pero se conocen las mediciones calorim\'etricas siguientes:

{\small \begin{tabular}{rclrr}
CH$_3$ COOH{\footnotesize (l)} +
$2$O$_2${\footnotesize (g)}&$\longrightarrow$ & 
$2$CO$_2${\footnotesize (g) +
$2$H$_2$O{\footnotesize (l)}}& $\Delta  H= -208.34$ kcal & (a) \\
C{\footnotesize (s)}+
O$_2${\footnotesize (g)}&
$\longrightarrow$&
CO$_2${\footnotesize (g)} & $\Delta  H= -94.05$ kcal &(b) \\
 H$_2${\footnotesize (g)}  + $\frac{1}{2}$O$_2${\footnotesize
(g)} &
$\longrightarrow$&
 H$_2$O{\footnotesize (l)}& $\Delta H= -68.32$ kcal &(c) \\ 
\end{tabular}}
\vskip .12in
Si ahora multiplicamos las ecuaciones (b) y (c) por 2 y las sumamos.
obtendremos:
\begin{center}
\begin{tabular}{rclrr}
$2$C{\footnotesize (s)}+
$2$O$_2${\footnotesize (g)}&
$\longrightarrow$&
$2$CO$_2${\footnotesize (g)} & $\Delta  H= -188.10$ kcal \\
 $2$H$_2${\footnotesize (g)}  + O$_2${\footnotesize
(g)} &
$\longrightarrow$&
 $2$H$_2$O{\footnotesize (l)}& $\Delta H= -136.64$ kcal & \\ \hline
$2$C{\footnotesize (s)}+$2$H$_2${\footnotesize (g)}  +
$3$O$_2${\footnotesize (g)} &
$\longrightarrow$&
$2$CO$_2${\footnotesize (g)} + $2$H$_2$O{\footnotesize (l)}&$\Delta 
H= -324.74$ kcal &(d) \\
\end{tabular}
\end{center}
y por sustracci\'on de la ecuaci\'on (a) en la (d) resulta:

%\begin{center}
%$2$C{\footnotesize (s)}$ + 2$H{\footnotesize (g)}$+$O$_2${\footnotesize(g)}$\;\longrightarrow\;$CH$_3$ COOH{\footnotesize (l)} \hskip.12in $\Delta  H =-116.40$ kcal
%\end{center}
\hskip0.25in\ce{2C (s) + 2H2 (g) + 2O2 (g) -> CH3COOH (l)} \hskip.12in $\Delta  H =-116.40$ kcal

\end{example}

\begin{exercises}
\exer Calcular la entalp\'{\i}a de reacci\'on a partir de las entalp\'{\i}as de enlace para las siguientes reacciones:
\subexer \ce{CH4 (g) + 2O{=}O (g) -> O{=}C{=}O (g) + 2H2O_{(g)}}

\subexer \ce{H2C=CH2 (g) + Br2 (g) -> H2BrC-CH2Br (g)}

\subexer\ce{HOC-H (g) + HO-OH (g) ->  HOC-OH + H2O (g)}

\exer Si se a\~naden 20g de agua a 20$\degreecelsius$ a 80 g de agua a 40 $\degreecelsius$ ?`Cu\'al es la temperatura final de la mezcla si el Cp del agua es 1 $\frac{\textrm{cal}}{\gram\degreecelsius}?$
\exer Con base en los siguientes datos a 25$\degreecelsius$:

{\centering
\vskip6pt
\begin{tabular}{rcll} 
$\frac{1}{2}H_2${\footnotesize (g)}
$+\frac{1}{2}O_2${\footnotesize (g)}  &
$\longrightarrow$ &
$OH${\footnotesize (g)},&
$\Delta H^\circ = $10.06 kcal,\\
$H_2${\footnotesize (g)}
$+\frac{1}{2}O_2${\footnotesize (g)} &
$\longrightarrow$ &
$H_2O${\footnotesize (g)},&
$\Delta H^\circ = $-57.80 kcal,\\ 
$ H_2${\footnotesize (g)} &
$\longrightarrow$ &
$2H${\footnotesize (g)},&
$\Delta H^\circ = $104.178 kcal,\\
$ O_2${\footnotesize (g)} &
$\longrightarrow$ &
$2O${\footnotesize (g)},&
$\Delta H^\circ = $118.318 kcal,\\
\end{tabular}
\vskip6pt
}
Calcular $\Delta H^\circ$ para:
\subexer \ce{OH -> H_{(g)} + O_{(g)}}
\subexer \ce{H2O -> 2H_{(g)} + O_{(g)} }
\subexer \ce{H2O -> H_{(g)} + OH_{(g)} }
\end{exercises}


\section[Entropia]{Entrop\'{\i}a}

Las transformaciones reales tiene una direcci\'on que consideramos natural. La transformaci\'on en sentido opuesto no ser\'{\i}a natural; ser\'{\i}a irreal. No podemos extraer calor del hielo para calentar el agua. La experiencia nos ense\~na  que tal transferencia de calor de una temperatura m\'as baja a otra mayor no se efect\'ua espont\'aneamente.

 En la primera ley de la termodin\'amica no se menciona nada relacionado con la direcci\'on que siguen los procesos. S\'olo exige que la energ\'{\i}a del universo permanezca igual antes y  despu\'es del proceso. Para lograr un trabajo por medio del calor, como en una m\'aquina t\'ermica, es esencial que exista una ca\'{\i}da de temperatura y que el calor fluya desde una tempe\-ratura elevada a otra menor. Pero a\'un bajo tales condiciones no todo calor se convierte en trabajo, sino s\'olo una fracci\'on del mismo, determinado en condiciones ideales por las temperaturas de operaci\'on. A\'un m\'as, aunque el calor de la m\'aquina permanezca inalterado en tal operaci\'on, el que permaneci\'o sin conversi\'on es degradado porque ha descendido la temperatura. De estos hechos puede verse que \textit{el calor no se transforma en trabajo sin producir cambios permanentes bien sea en los sistemas o en sus proximidades.}

Para poder especificar el sentido que llega a seguir un sistema definamos una nueva cantidad matem\'atica $S$, denominada \textit{\gloss[word]{entropia}} del sistema. La entrop\'{\i}a \index{entrop\'{\i}a} de un sistema s\'olo depende de sus estados inicial y final.

Pero ?`qu\'e es la entrop\'{\i}a?, en primer lugar hay que considerar que la entrop\'{\i}a no es alguna cosa que se puede ver o sentir o colocar en una botella con s\'olo observar el sistema desde un \'angulo apropiado. La entrop\'{\i}a es algo menos palpable que una cantidad de calor o de trabajo. Es mejor preguntarse ?`c\'omo varia la entrop\'{\i}a con la temperatura a presi\'on constante?, ?`c\'omo varia con el volumen a temperatura constante? Si conocemos el comportamiento de la entrop\'{\i}a bajo diferentes circunstancias, sabremos algo acerca de su naturaleza.
Por definici\'on la entrop\'{\i}a se expresa de la siguiente manera:
   \begin{equation}
   \Delta S = \frac{Q _{\mathrm{rev}}}{T}
   \end{equation}
Donde $Q _{\mathrm{rev}}$ es una cantidad de calor absorbido durante un proceso reversible que ocurre a
la temperatura absoluta  $T$.

Cuando $Q _{\mathrm{rev}}$ es positivo, es decir hay absorci\'on de calor, entonces tambi\'en \gloss[word]{deltas} es positiva, indicando un incremento en la entrop\'{\i}a del sistema. Por otra parte cuando hay desprendimiento de calor  $Q _{\mathrm{rev}}$ es negativa y lo es $\Delta S$, y el
sistema experimenta un decremento entr\'opico. Las entrop\'{\i}as y cambios entr\'opicos se expresan en \index{entrop\'{\i}a!unidades} calor\'{\i}as por grado para una cantidad de sustancia dada. La cantidad \textit{calor\'{\i}a por grado} se denomina \textit{unidad
entr\'opica} (ue).

La \textbf{segunda} ley de la termodin\'amica\index{termodinamica@termodin\'amica!segunda Ley} nos dice,
que \textit{todo proceso natural se verifica con un incremento entr\'opico y que la
direcci\'on del cambio es aquella que conduce a tal aumento}. Nos indica la cantidad de energ\'{\i}a no utilizable contenida en un sistema.

\begin{enumerate}
\item En cualquier proceso o ciclo \textit{reversible} $\Delta S = 0$
\item En cualquier otro proceso o ciclo \textit{irreversible} $\Delta S > 0$
\end{enumerate}

\paragraph{Cambios de entrop\'{\i}a para gases ideales tenemos} \index{gases
ideales!entrop\'{\i}a} A partir de la definici\'on de entrop\'{\i}a y la ecuaci\'on general de
los gases $PV = nRT$ tenemos:

\begin{equation}
\Delta S = n C_p \ln \frac{T_2}{T_1} -nR\ln \frac{P_2}{P_1}
\end{equation}

\paragraph{Cambios de entrop\'{\i}a en las transformaciones f\'{\i}sicas.}
\index{entrop\'{\i}a!cambios}
 Se producen no s\'olo por variaci\'on de temperatura, presi\'on o volumen de un sistema,
sino por transformaciones f\'{\i}sicas tales como fusi\'on, vaporizaci\'on o transformaci\'on
de un forma cristalina a otra. Tales procesos tienen lugar reversiblemente a $T$ y $P$
constantes y van acompa\~nadas con una evoluci\'on o absorci\'on de calor $\Delta H$
calor\'{\i}as para una sustancia dada. Por eso  para tal proceso resulta:
\begin{equation}
\Delta S =  \frac{Q_{rev}}{T} = \frac{\Delta H}{T}
\end{equation}
\begin{example}
 Encontrar la diferencia de entrop\'{\i}a para la transici\'on:
 
\begin{tabular}{rcll}
H$_2$O{\small (L,1 atm)}  & $\longrightarrow$&H$_2$O{\small (g,1 atm)}&  $\Delta H_{373.2
K}= 9,717$ cal/mol\\
\end{tabular}

Como a 373.2 K y 1 atm de presi\'on el punto normal de ebullici\'on del agua  H$_2$O$_{(l)}$
est\'a en equilibrio con el  H$_2$O$_{(g)}$, entonces
\vskip.15in
\hskip 1.5in $\Delta S = S_g - S_l = \frac{\Delta H}{T} $

\hskip 1.8in $ = \frac {9,717}{373.2}$

\hskip 1.8in $ = 26.04 \textrm{ ue}/\mole$ 
\end{example}

\subsection{Cambios de entrop\'{\i}a en las reacciones qu\'{\i}micas}
Para una reacci\'on cual\-quie\-ra tal como
\begin{equation}
aA +bB +\ldots = cC + dD + \ldots
\end{equation}
el cambio entr\'opico esta dado por
\begin{equation}
\Delta S = (cS_C + dS_D + \ldots) - (aS_A + bS_B + \ldots)
\end{equation}
En nuestro caso s\'olo vamos a estudiar las reacciones qu\'{\i}micas a tem\-pe\-ratura y presi\'on constantes

\begin{table} [ht]
\begin{minipage}{\linewidth}
\caption[Entrop\'{\i}a absoluta]{Entrop\'{\i}a\footnote{Unidades entr\'opicas (ue) por
mol. Fuente:  Maron \& Prutton 1990} absoluta  de  elementos y compuestos a 25$\degreecelsius$}
\begin{center}
{\small \begin{tabular}{||ll|ll|ll||} \hline
\textbf{Sustancia}&\textbf{$S ^\circ$}&\textbf{Sustancia}&\textbf{$S^\circ$}
&\textbf{Sustancia}&\textbf{$S^\circ$}\\ \hline
 H$_{2(g)}$        & 31.21  & H$_2$O$_{(l)}$  & 16.72 & Na$_{(s)}$  & 12.2 \\ 
 C$_{(\mathrm{Diamante})}$  & 0.583  & H$_2$O$_{(g)}$  & 45.11 & Mg$_{(s)}$  & 7.77 \\ 
 C$_{(\mathrm{Grafito})}$   & 1.36   & CO$_{(g)}$      & 47.30 & Cl$_{2(g)}$ & 53.29\\ 
 N$_{2(g)}$        & 45.77  & CO$_{2(g)}$     & 51.06 & AgCl$_{(s)}$& 22.97\\
 O$_{2(g)}$        & 49.00  & HgCl$_{2(s)}$   & 34.6  & NaCl$_{(s)}$& 17.3 \\
 Fe$_{(s)}$        & 6.49   & Fe$_2$O$_{3(s)}$& 21.5  & Ag$_{(s)}$  & 10.21\\
 Hg$_{(l)}$        & 18.5   & Hg$_{(g)}$      & 41.80 & HgCl$_{(s)}$ & 23.5\\
 C$_2$H$_{6(g)}$   & 54.85  & CH$_3$OH$_{(l)}$& 30.2  & C$_6$H$_5$OH$_{(l)}$& 34.0\\
 C$_2$H$_5$OH$_{(l)}$&38.4  & C$_6$H$_{6(l)}$ & 41.30 & CH$_3$COOH$_{(l)}$&38.2 \\
 \hline 
\end{tabular}}
\label{tab:4}
\end{center}
\end{minipage}
\end{table}

\begin{example}
 Supongamos que se busca el cambio de entrop\'{\i}a necesario para la reacci\'on:
\begin{center}
C$_{(\mathrm{s,grafito})}$ + 2H$_2$ + $\frac{1}{2}$O$_{2(g)}$ $\longrightarrow$  CH$_3$OH$_{(l)}$
\end{center}
Empleando la entrop\'{\i}a molar presentada en el \textbf{Cuadro~\ref{tab:4}}, obtenemos:

\begin{tabular}{rcrcrcrcrc}
$\Delta S^\circ _T$ &$=$&$S^\circ _{CH_2OH}$&$-$&$(S^\circ
_C$&$+$&$2S^\circ_{H_2}$&+&$\frac{1}{2} S^\circ_{O_2})$\\
&$=$ &$30.3$ & $-$& $1.36$ &$-$&$62.42$&$-$&$24.50$\\
&$=$ &$-58.0 $ue&&&&&&\\
\end{tabular}

De igual manera se calcula $\Delta S^\circ$ de otras reacciones a 25$\degreecelsius$ con tal de que
se conozcan las entrop\'{\i}as absolutas necesarias.
\end{example}

\subsection{Energ\'{\i}a libre y espontaneidad}
El valor de la entrop\'{\i}a es cero \'unicamente en el caso de un cristal per\-fec\-to a una temperatura de cero Kelvin. Esto se conoce como la \textbf{tercera} ley de la termodin\'amica \index{termodinamica@termodin\'amica!tercera ley de la}. La entrop\'{\i}a aumenta con la temperatura y
disminuye con la mis\-ma. En resumen:\\
\begin{center}
\begin{tabular}{cc}
\textbf{Reacci\'on espont\'anea} & \textbf{Reacci\'on no espont\'anea }\\
$\Delta H = -$ & $\Delta  H = +$\\
$\Delta S = +$ & $\Delta  S = -$ \\
\end{tabular}
\end{center}
La energ\'{\i}a libre de Gibbs cuantifica la espontaneidad de una reacci\'on a partir de sus
valores de $\Delta H$ y $\Delta S$ as\'{\i} tenemos que:
\begin{equation}
\Delta G = \Delta H - T\Delta S
\label{eq:6}
\end{equation}
\begin{tabular}{lcl}
donde:&\\
&$\Delta G$& cambio en la energ\'{\i}a libre de Gibbs \\
&$\Delta H$ &cambio en la entalp\'{\i}a\\
&$T$ &temperatura absoluta.\\
&$\Delta S$ &cambio en la entrop\'{\i}a\\
\end{tabular}

El criterio de espontaneidad es la disminuci\'on en la energ\'{\i}a libre de Gibbs lo anterior indica que 
\gloss[word]{deltag}$ = -$.\index{energ\'{\i}a!libre de Gibbs}

Pregunta ?`en una reacci\'on endot\'ermica en la que adem\'as disminuye la entrop\'{\i}a es
espont\'anea?\\ 
?`C\'omo explican la reacci\'on exot\'ermica como la combusti\'on de la gasolina?.
La combusti\'on ocurre \index{proceso!espont\'aneo} espont\'aneamente pues los productos de combusti\'on tienen una energ\'{\i}a menor que la gasolina y el ox\'{\i}geno. La entrop\'{\i}a de los gases de combusti\'on calientes, al difundirse en el ambiente dispersan e\-nerg\'{\i}a por lo que la entrop\'{\i}a del universo aumenta. Los gases de com\-busti\'on \textit{solitos} no pueden reaccionar espont\'aneamente para regenerar gasolina y ox\'{\i}geno porque \'esto en lugar de aumentar la entrop\'{\i}a del universo la disminuir\'{\i}a, violando la segunda ley de la termodin\'amica (aunque se respete la primera).

Un proceso no espont\'aneo\index{proceso!no espont\'aneo} puede ocurrir a costa de un mayor desorden en la entrop\'{\i}a de otro sistema.

En un refrigerador. La temperatura en el interior disminuye y con ella la entrop\'{\i}a pero
a cambio el calor dispersado por el motor en el medio es mucho mayor que el absorbido por el
refrigerador, por lo que globalmente se cumple la segunda ley.
\begin{example}
La oxidaci\'on del hierro es un proceso exot\'ermico que ocurre con aumento de entrop\'{\i}a
sin embargo es tan lento que se necesitan meses para observar alg\'un efecto.
\vskip .2in
2 Fe$_{(s)}$ + $\frac{3}{2}$O$_{2(g)}$ $\longrightarrow$ Fe$_2$O$_{3(s)}$ \hskip.5in $\Delta
H^\circ = - 196,000 cal$
\vskip .2in
$\Delta$ S$ = \Delta$ S$^\circ _{Fe_2O_{3(s)}} - (2 \Delta S^\circ _{Fe _{(s)}} +
\frac{3}{2}  \Delta S^\circ _{O _{2(g)}} )$

$\Delta$ S $= 21.5 - (2 (6.49) + \frac{3}{2} 49.0) = 21.5 -12.98 - 73.5$

$\Delta$ S $= - 64.98 ue/mol$

A partir de los datos anteriores se puede calcular la $\Delta G$ sustituyendo en la
ecuaci\'on~\ref{eq:6} tenemos:

$\Delta$ G $= - 196,000 cal - 298 (- 64.98 \frac{cal}{mol K})$

$\Delta$ G $=  -176,635.96 cal$

A partir de este valor podemos decir que esta reacci\'on es espont\'anea.


\end{example}

\subsection{Energ\'{\i}a en los seres vivos}

El adenos\'{\i}n trifosfato (\gloss[short]{atp}) \index{ATP} es la unidad biol\'ogica universal de energ\'{\i}a, es la mol\'ecula que libera energ\'{\i}a  al romperse formando el denos\'{\i}n difosfato (ADP)  y es el lazo principal entre las actividades celulares que rinden energ\'{\i}a y las que la consumen.

\begin{equation}
\ce{C6(H2O)+ O2 ->  6CO2 + 6H2O }
\end{equation}
Para la reacci\'on anterior tenemos la siguiente informaci\'on:

\begin{tabular}{rcl}
$\Delta G$ & = & $-686,000 $ cal/mol \\
$\Delta H$ & = & $-673,000 $ cal/mol \\
\end{tabular}

Con estos datos podemos saber si esta reacci\'on es espont\'anea o no para lo cual empleamos
la Ecuaci\'on~\ref{eq:6} y despejando $\Delta S$ obtenemos:

$$\Delta S = \frac{\Delta H-\Delta G}{T} =\frac{-673,000 \textrm{ cal/mol}-( -686,000 \textrm{ cal/mol})}{298
K}= 44 \frac{\mathrm{cal}}{\textrm{mol K}}$$

Como $\Delta S$ es positiva y $\Delta G$ es negativa decimos que esta reacci\'on de
oxidaci\'on de la glucosa es espont\'anea.

\subsection{Reacciones enderg\'onicas y exerg\'onicas}
En la \textbf{Figura~\ref{atp}} se muestra c\'omo la oxidaci\'on de mol\'eculas ayudan a sintetizar el
\gloss[short]{atp} que es una mol\'ecula altamente energ\'etica que se emplea para efectuar
diversos procesos en los organismos vivos form\'andose el \gloss[short]{adp} m\'as un
fosfato libre $P_i$

\begin{figure}[bht]
\begin{picture}(130,30)(-8,0)
\put(3,0){\scriptsize O$_2$}
\put(0,25){\scriptsize CO$_2$}
\put(37,0){\scriptsize$ADP + P_i$}
\put(40,25){\scriptsize$ATP$}
\put(10,8){\shortstack{{\scriptsize Oxidaci\'on de}\\ {\scriptsize mol\'eculas}\\
{\scriptsize combustibles que}\\ {\scriptsize producen energ\'{\i}a} }}
\put( 55,8){\shortstack{{\scriptsize Bios\'{\i}ntesis}\\  {\scriptsize (Trabajo}\\ {\scriptsize
bioqu\'{\i}mico)}}}
\put( 80,8){\shortstack{{\scriptsize Transporte}\\ {\scriptsize Activo}\\ {\scriptsize
(Trabajo}\\ {\scriptsize Osm\'otico)}}}
\put(100,8){\shortstack{{\scriptsize Contracci\'on}\\ {\scriptsize muscular}\\ {\scriptsize
(Trabajo}\\ {\scriptsize mec\'anico)}}}
\thicklines
%  Flechas del CO2 y del O2
\put(9,26){\vector(-1,0){1}}
\qbezier(17,22)(15,26)(9,26)
\put(17,6){\vector(2,3){1}}
\qbezier(9,1)(15,1)(17,6)
%  Flechas al ATP y del ADP + Pi
\put(36,26){\vector(1,0){1}}
\qbezier(25,22)(28,26)(36,26)
\put(25,6){\vector(-2,3){1}}
\qbezier(36,1)(28,1)(25,6)
%  Flecha superior
\put(50,26){\line(1,0){45}}
\qbezier(50,26)(58,26)(60.5,19.5)
\put(60.5,20){\vector(2,-3){1}}
\qbezier(75,26)(83,26)(85.5,23)
\put(85.5,23){\vector(2,-3){1}}
\qbezier(95,26)(103,26)(105.5,23)
\put(105.5,23){\vector(2,-3){1}}
%  Flecha inferior
\put(97,1){\vector(-1,0){43}}
\qbezier(106,7)(104,1)(97,1)
\put(77,1){\vector(-1,0){1}}
\qbezier( 86,7)(84,1)(77,1)
\put(97,1){\vector(-1,0){1}}
\qbezier( 63,7)(61,1)(54,1)
\end{picture}
\caption{Ciclo energ\'etico. Formaci\'on de ATP y su empleo}
\label{atp}
\end{figure}
As\'{\i} dentro de los diferentes procesos del metabolismo de los seres vivos se efect\'uan reacciones  donde se almacena energ\'{\i}a a partir de los  alimentos para ser luego empleada seg\'un se requiera dentro del organismo.

\subsubsection{\gloss[Word]{endergonicas}}\index{reaccion@reacci\'on!enderg\'onica} Aquellas
reacciones bioqu\'{\i}micas que para ocurrir requieren de energ\'{\i}a proveniente de enlaces
de alta energ\'{\i}a como el ATP.

\subsubsection{\gloss[Word]{exergonicas}}\index{reaccion@reacci\'on!exerg\'onica} Aquellas reacciones
bioqu\'{\i}micas que almacenan la energ\'{\i}a en enlaces de alta energ\'{\i}a como en las
mol\'eculas de ATP.

\newpage
\begin{exercises}

\exer Si para la reacci\'on A + B
$\longrightarrow$ C la
$\Delta H = -2,500 \mathrm{cal}/\mole$ a   T $= 500 \kelvin$. Calcular la entrop\'{\i}a de la reacci\'on.
\exer Usando la ecuaci\'on de energ\'{\i}a libre de
Gibbs calcular la
$\Delta H$ para la si\-guiente reacci\'on:

\reaction*{C_{(s,grafito)}  + O2 -> CO2_{(g)} } 

si $\Delta G^\circ_{298K} = -94.3 kcal$ y la $\Delta S$ la puede calcular a partir de la
\textbf{Cuadro~\ref{tab:4}} (p\'agina \pageref{tab:4}).

\exer A partir de las siguientes reacciones:
\begin{description}
\item[(a)] \ce{CO_{(g)} + 2H_{2(g)}  -> CH_3OH_{(l)}} \hskip  .7in $\Delta H^\circ=-30.61 kcal/\mole$
\item[(b)] \ce{2HgCl_{2(s)} ->  2Hg_{(l)} +Cl_{2(g)}}\hskip  .70in $\Delta H^\circ= 448.44 \kilo\joule/\mole $
\item[(c)]  \ce{MgO_{(s)} + H_{2(g)} ->H_2O_{(l)} + Mg_{(s)}} \hskip  .38in $\Delta H^\circ= 315\kilo\joule/\mole $
\end{description}
 \subexer  Con los datos del Cuadro~\ref{tab:4}  hallar los cambios de entrop\'{\i}a tipo que acompa\~nan a las reacciones
 a T$= 25\degreecelsius$.\\
\subexer A partir de los datos de $\Delta H$ y $\Delta S$
calcular la $\Delta G$ para cada reacci\'on y decir si son espont\'aneas o no.
 
\exer Calcule $\Delta S$ para los siguientes casos
(T=25$\degreecelsius$):\\
\begin{tabular}{cll}
a)&$\Delta H^\circ = 13.23 \textrm{ kcal}$ &$\Delta G^\circ=5.82\textrm{ kcal}$\\
 b)&$\Delta H^\circ = -75.18 \textrm{ kcal}$ &$\Delta G^\circ=66.71 \textrm{ kcal}$\\
c)&$\Delta H^\circ = 33.82 \textrm{ kcal}$&$\Delta G^\circ=-24.60 \textrm{ kcal}$\\
\end{tabular}
\end{exercises}


