\chapter{Velocidad y equilibrio}\index{velocidad}

Se define la velocidad de la reacci\'on qu\'{\i}mica. Mediante la teor\'{\i}a de las co\-lisiones se estudia y explica el mecanismo de reacci\'on: el perfil energ\'etico, el efecto de algunos de los factores como la concentraci\'on, la temperatura, la superficie de contacto y los catalizadores sobre la velocidad de reacci\'on.
\section{Velocidad de la reacci\'on qu\'{\i}mica}
En una explosi\'on de un petardo el cambio qu\'{\i}mico es tan r\'apido y violento que en un instante ocurre. Si cortas una manzana, o la rallas, en pocos minutos observas que la superficie se obscurece. Este cambio qu\'{\i}mico es m\'as lento que la explosi\'on. Si ahora sostienes un objeto de hierro (sin recubri\-miento) seguramente te cansar\'as antes de que se observe alg\'un cambio como la formaci\'on de una capa de \'oxido sobre su superficie.

Lo anterior ejemplifica que las reacciones no tardan el mismo tiempo en ocurrir. Existen cambios qu\'{\i}micos que se verifican instant\'aneamente; otros son un poco m\'as lentos, tardan minutos, y otros, son tan lentos que tardan meses o a\~nos. 

Cuando el qu\'{\i}mico desea conocer la velocidad de las reacciones qu\'{\i}micas aplica la cin\'etica qu\'{\i}mica.\\ 
La cin\'etica qu\'{\i}mica \index{cin\'etica qu\'{\i}mica} es la parte de la qu\'{\i}mica que estudia la velocidad de la reacci\'on, los mecanismos o etapas por medio de los cuales los reactivos se convierten en productos.

De esta forma la cin\'etica es un complemento de la termodin\'amica al proporcionar informaci\'on de la velocidad y mecanismos de transformaci\'on de reactivos en los productos.

 La simple diferencia entre reacciones lentas y r\'apidas no pasa de ser una curiosidad, una pregunta interesante es ?`c\'omo hacer que una reacci\'on determinada sea m\'as r\'apida o lenta?.

En la industria se tiene algunas ventajas al aumentar la velocidad de reacci\'on y obtener una tonelada de producto en 2 minutos en vez de 8 horas.

En el hogar se aprecia la ventaja de reducir la velocidad de descomposici\'on de los alimentos para que en lugar de pudrirse en unas horas duren varios d\'{\i}as o hasta meses.

\subsection{Definici\'on de velocidad qu\'{\i}mica}
Es el tiempo que tardan los reactivos en convertirse en productos, bajo ciertas circunstancias.

Si consideramos la unidad de tiempo (por ejemplo el segundo) la velocidad de reacci\'on la podemos entender como las moles de reactivo que se convierten en producto en dicha unidad. Por ejemplo el cambio qu\'{\i}mi\-co en el que, en un volumen de un litro, reaccionan 20 moles de reactivo cada  segundo tiene mayor velocidad que cuando reacciona una \mole en el mismo lapso.

\subsection{Teor\'{\i}a de las colisiones}
El enfoque molecular de los cambios qu\'{\i}micos nos permite explicarlos como una transformaci\'on molecular de la materia.

En el modelo de reacci\'on qu\'{\i}mica que se ha manejado el  cambio resulta de la ruptura de uno o varios enlaces en las  mol\'eculas de los reactivos, el intercambio de  part\'{\i}culas y la formaci\'on de nuevos enlaces que dan lugar a los productos. Se especific\'o mejor este modelo al aplicarle la ley de la conservaci\'on de la masa (balance de ecuaciones): el total de \'atomos de cada elemento antes y despu\'es del intercambio es constante. En la estequiometr\'{\i}a relacionamos este modelo con la escala macrosc\'opica mediante el concepto de mol. En termodin\'amica estudiamos la energ\'{\i}a que se manifiesta en las reacciones, \'esta se  encuentra en los enlaces mole\-culares e intuimos la idea de que las mol\'eculas no est\'an quietas sino en un movimiento que da lugar  a la temperatura.
 
Sin embargo ?`c\'omo ocurren las reacciones? para responder a \'esta recurriremos a lo que se ha visto hasta ahora. Si no hay contacto o choque no ocurre la reacci\'on. En el caso de dos s\'olidos las part\'{\i}culas que los constituyen est\'an fijas en una red cristalina y sus movimientos son principalmente de vibraci\'on. La ausencia de  movimientos que acerquen a las mol\'eculas explica el porqu\'e los s\'olidos no reaccionan, a menos que se di\-suelvan o fundan. Cuando las mol\'eculas de los reactivos son gases, o est\'an disueltas en agua, pueden acercarse y colisionar.

No todos las colisiones  producen reacci\'on. Si todos los
choques fueran efectivos las reacciones de gases ser\'{\i}an
instant\'aneas, lo que no se observa experimentalmente.

La teor\'{\i}a de las colisiones explica que los choques  fruct\'{\i}feros son aquellos en los que:
\begin{enumerate}
\item Las mol\'eculas tienen una energ\'{\i}a mayor que la
energ\'{\i}a de activaci\'on.
\item La orientaci\'on es la adecuada.
\end{enumerate}
Si las mol\'eculas tienen poca energ\'{\i}a chocan pero los enlaces de los reactivos no se rompen y la colisi\'on no es fruct\'{\i}fera. Si la velocidad del choque corresponde a una energ\'{\i}a mayor que la de activaci\'on, se forman nuevos enlaces, hay un intercambio de \'atomos y luego se separan nuevas mol\'e\-culas.

El intercambio de part\'{\i}culas resulta de la ruptura y formaci\'on simult\'a\-nea de enlaces en el complejo
activado y no de un movimiento real de las part\'{\i}culas.

La especie intermedia se llama \textbf{\gloss{complejoactivado}}
y s\'olo existe durante una fracci\'on de segundo, puede dar
lugar a los productos o a los reactivos.

Por otro lado si la orientaci\'on no es la adecuada, no existe un intercambio de part\'{\i}culas y cada mol\'ecula se aleja inalterada.

A partir de lo anterior el modelo de la teor\'{\i}a de las colisiones establece que cualquier factor que incremente la frecuencia de choque entre las mo\-l\'e\-culas de reactivos aumenta la velocidad de reacci\'on, y viceversa.

La teor�a de colisiones es aplicable s�lo a reacciones bimolecular elementales. Como ejemplo, seleccionadnos una reacci�n elemental del tipo:

$$\ce{A + B -> C + D}$$

Es claro que esta reacci�n no puede producirse m�s r�pidamente que el n�mero de veces que chocan las mol�culas A y B. El n�mero de colisiones entre las mol�culas A y B en un \centi\square\metre\per\second, esta dada por la ecuaci�n:
\begin{equation}
 Z_{AB}= \left( \frac{ \sigma_A+ \sigma_B }{2}\right)^2 \sqrt[2]{\frac{8 \pi(m_a+m_b)\textbf{k}T }{m_Am_B}} n_An_B 
\end{equation}
donde $\sigma_A$ y $\sigma_B$  son los di�metros moleculares, $m_A$ y $m_B$ las masa moleculares, y $n_A$ y $n_B$ el n�mero de mol�culas de A y B en un \centi\cubic\metre. Si en cada colisi�n hay reacci�n, entonces esto ser�a igual a la velocidad de consumo de A o de B:
\begin{equation}
 -\frac{\mathrm{d}n_A}{\mathrm{d}t}=-\frac{\mathrm{d}n_B}{\mathrm{d}t}=\left(\frac{ \sigma_A+ \sigma_B }{ 2 }\right)^2 \sqrt{\frac{8 \pi(m_A+m_B)\textbf{k}T }{m_Am_B}}n_An_B 
\end{equation}


Sin embargo, toda colisi�n no conduce a una reacci�n entre A y B, s�lo aquellas donde la energ�a de las mol�culas excede el valor de E* . La fracci�n de colisiones en donde esto ocurre es proporcional a \( \exp^{-(E/RT)} \), as�  la velocidad de reacci�n es
\begin{equation}
 -\frac{\mathrm{d}n_A}{\mathrm{d}t}=\left(\frac{ \sigma_A+ \sigma_B }{ 2 }\right)^2 \sqrt{\frac{8 \pi(m_A+m_B)\textbf{k}T }{m_Am_B}} \exp^{-(E/RT)} n_An_B 
\end{equation}
La ecuaci�n emp�rica para la velocidad de la reacci�n elemental es \( -\mathrm{d}n_a/\mathrm{d}t=kn_An_B \), as� se obtiene la constante de velocidad

\begin{eqnarray}
 k&=&\left(\frac{ \sigma_A+ \sigma_B }{ 2 }\right)^2 \sqrt{\frac{8 \pi(m_A+m_B)\textbf{k}T }{m_Am_B}} \exp^{(-\frac{E}{RT})} \\
 k&=&Z'e^{-E/RT}
\end{eqnarray}

donde \( Z'=Z_{AB}/n_An_B \)

La ecuaci�n de Arrhenius \index{Arrhenius} tiene la misma forma de la ecuaci�n anterior, por lo tanto la teor�a de las colisiones predice para el factor de frecuencia 

\begin{equation}
A=Z'= \left(\frac{ \sigma_A+ \sigma_B }{ 2 }\right)^2 \sqrt{\frac{8 \pi(m_A+m_B)\textbf{k}T }{m_Am_B}}
\end{equation}

El orden de magnitud de A para reacciones bimoleculares es de $10^9$ a $10^{10}$ para unidades de \mole\per\liter~y temperatura de 300\kelvin.

La teor�a de colisi�n predice satisfactoriamente el valor de la constante de velocidad cuando se tienen mol�culas simples, siempre que se conozca la energ�a de activaci�n. Para mol�culas complejas las velocidades tienden a ser m�s peque�as, debido a que  colisiones que poseen la energ�a exigida, es posible que no produzcan reacci�n ya que no colisionan en la posici�n adecuada.

\subsection{Energ\'{\i}a de activaci\'on}

Una barrera de energ\'{\i}a separa el estado de los reactivos y el estado de los productos. Al chocar, los reactivos deben tener energ\'{\i}a suficiente para  vencer esta barrera, de modo que se formen los productos. La altura de la barrera es $H^* - H^\circ _R$; esta es la \gloss[word]{energiadeactivacion}.
\index{energ\'{\i}a! de activaci\'on}  En la \textbf{Figura~\ref{e-activa}} se representa el proceso de reacci\'on. 

\begin{figure}
\begin{picture}(60,50)
\put(18,40){\footnotesize \textbf{Reacci\'on endot\'ermica}}
\put(5,5){\vector(1,0){55}}
\put(5,5){\vector(0,1){35}}
\put(20,1){\scriptsize Avance de reacci\'on}
\put(-1,32){\scriptsize $\Delta H$}
\thicklines
\put(7,10){\line(1,0){13}}
\qbezier(20,10)(23,10)(25,15)
\qbezier(25,15)(30,55)(35,23)
\qbezier(35,23)(37,18)(40,18)
\put(40,18){\line(1,0){15}}
\thinlines
\put(9,11){\scriptsize Reactivos}
\put(41,19){\scriptsize Productos}
%lineas a la energia de activacion
\begin{dottedjoin}{0.5}
\jput(30,37){}\jput(45,37){}
\end{dottedjoin}
\begin{dottedjoin}{0.5}
\jput(20,10){}\jput(55,10){}
\end{dottedjoin}
\put(35,30){\vector(0,1){7}}
\put(35,30){\vector(0,-1){20}}
\put(36,26){\shortstack[l]{{\tiny Energ\'{\i}a de activaci\'on}\\[-2.5pt]
 {\tiny empleada para iniciar}\\[-2.5pt] {\tiny la reacci\'on}}}
% Lineas de energia absorbida
\put(40,13){\vector(0,1){5}}
\put(40,13){\vector(0,-1){3}}
\put(41,12){\shortstack[l]{{\tiny Energ\'{\i}a absorbida}\\[-2.5pt]
 {\tiny por la reacci\'on}}}
\end{picture}
\begin{picture}(60,50)
\put(5,5){\vector(1,0){55}}
\put(5,5){\vector(0,1){35}}
\put(23,40){\footnotesize \textbf{Reacci\'on exot\'ermica}}
\put(20,1){\scriptsize Avance de reacci\'on}
\put(-1,32){\scriptsize $\Delta H$}
\thicklines
\put(7,29){\line(1,0){13}}
\qbezier(20,29)(23,29)(25,34)
\qbezier(25,34)(32,48)(35,15)
\qbezier(35,15)(37,10)(40,10)
\put(40,10){\line(1,0){15}}
\thinlines
\put(7,30){\footnotesize Reactivos}
\put(43,11){\footnotesize Productos}
% Lineas a la energia de activacion
\begin{dottedjoin}{0.5}
\jput(30,38){}\jput(41,38){}
\end{dottedjoin}
\begin{dottedjoin}{0.5}
\jput(7,10){}\jput(40,10){}
\end{dottedjoin}
\put(41,30){\vector(0,1){8}}
\put(41,30){\vector(0,-1){20}}
\put(42,25){\shortstack[l]{{\tiny Energ\'{\i}a de activaci\'on}\\[-2.5pt]
 {\tiny empleada para iniciar}\\[-2.5pt] {\tiny la reacci\'on}}}
% Lineas de energia absorbida
\put(12,20){\vector(0,1){9}}
\put(12,20){\vector(0,-1){10}}
\put(13,17){\shortstack[l]{{\tiny Energ\'{\i}a}\\[-2.5pt] {\tiny
desprendida}\\[-2.5pt] {\tiny por la reacci\'on}}}
\end{picture}
\caption[Energ\'{\i}a de Activaci\'on]{Cambios de energ\'{\i}a en las
reacciones exot\'ermicas y endot\'ermicas}
\label{e-activa}
\end{figure}


En otras palabras es la energ\'{\i}a necesaria para iniciar una reacci\'on qu\'{\i}\-mica.

\subsection{Perfil de energ\'{\i}a}
Una vez que se ha suministrado la energ\'{\i}a de activaci\'on, se genera la suficiente energ\'{\i}a para mantener la reacci\'on en proceso.

Obs\'ervese que existen reacciones exot\'ermicas que s\'olo necesitan calor (e\-ner\-g\'{\i}\-a de activaci\'on) para iniciarse, como un proceso exot\'ermico. La com\-bus\-ti\'on del magnesio es altamente \gloss[word]{exotermico}, y sin embargo, el magnesio se debe calentar a una temperatura bastante alta para que comience su combusti\'on.  Una vez iniciada, la reacci\'on de combusti\'on   se lleva a cabo muy vigorosamente hasta que se termina el magnesio o el sumi\-nistro de ox\'{\i}geno. La descomposici\'on electrol\'{\i}tica del agua en hidr\'ogeno  ox\'{\i}geno es altamente endot\'ermica. Requiere de energ\'{\i}a  para llevarse a cabo.

\subsection[Factores que influyen en la velocidad]{Factores que influyen en la velocidad de las reacciones}
\index{reaccion@reacci\'on!velocidad}
Las fuerzas intermoleculares dicen si una reacci\'on es posible o no. Ahora, con ayuda de enfoque molecular de la teor\'{\i}a de las colisiones, podemos analizar cualitativamente las fuerzas cin\'eticas que mueven a una reacci\'on qu\'{\i}mica en el tiempo. Estas fuerzas cin\'eticas son los factores que mo\-di\-fican la velocidad de una reacci\'on. Experimentalmente se ha encontrado que estos factores son:

\subsubsection{Concentraci\'on}
En general la velocidad de una reacci\'on \index{velocidad de una reacci\'on!concentraci\'on} aumenta cuando se incrementa la concentraci\'on (moles/litro) de uno o m\'as reactivos.

De manera formal esto se expresa: \textit{la velocidad de una reacci\'on es directamente proporcional  a la concentraci\'on de los reactivos.} 
\subsubsection{Temperatura}
La velocidad de reacci\'on aumenta con la temperatura.
\index{velocidad de una reacci\'on!temperatura} 
Una regla emp\'{\i}rica muy \'util nos indica que por cada 10$^\circ$C que se incrementa la temperatura la velocidad se duplica (cocci\'on de alimentos).  Lo opuesto tambi\'en es v\'alido. Al bajar la temperatura, la velocidad dismi\-nuye (congelaci\'on de alimentos para conservarlos).

\subsubsection{Superficie de contacto}
La velocidad de reacci\'on  en la que interviene un s\'olido aumenta con el \'area superficial. La superficie de un s\'olido aumenta al reducir el tama\~no de part\'{\i}cula.
\subsubsection{Catalizadores}\index{catalizador}
Un \gloss[word]{catalizador} es aquella sustancia que modifica la
velocidad de reacci\'on y no sufre cambio significativo. La acci\'on del catalizador puede incrementar o disminuir la velocidad de reacci\'on. Si aumenta la velocidad\index{velocidad de una reacci\'on!catalizador} la sustancia se llama catalizador positivo, pero si la disminuye se llama catalizador negativo o inhibidor.
 
% \pagebreak
 
\begin{example}
En la preparaci\'on de ox�geno en el laboratorio, se usa di\'oxido de manganeso como catalizador para aumentar las velocidades de descomposici�n, tanto del clorato de potasio, como del per\'oxido de hidr\'ogeno:

$$ \ce{2KClO3 (ac) ->[\ce{MnO2}]  2KCl (s)  + 3O2 (g)} $$
%$$ \ce{2H2O_{2(ac)}} \buildrel \ce{MnO2}\over\longrightarrow\ce{2 H_2O_{(l)} + O_{2(g)}}$$
$$\ce{2H2O2 (ac) ->[\ce{MnO2}]  2H2O (l) + O2 (g)} $$
\end{example}

\subsubsection{Naturaleza de los reactivos}

Los compuestos con enlace i\'onico en disoluci\'on acuosa tienen mayor velocidad de reacci\'on que los compuestos covalentes.\index{velocidad de una reacci\'on!naturaleza de los reactivos}

Adem\'as de los factores anteriores hay otros que tambi\'en modifican la velocidad de una reacci\'on.

\begin{itemize}
\item La presi\'on para reacciones en estado  gaseoso.
\item La presencia de luz ultravioleta en algunas reacciones como la halogenaci\'on de alcanos. 
\end{itemize}

