\chapter*{Anexo}

\section {Demostraci\'on de pH de lluvia \'acida}


Conciderando que el agua tambien tiene un equilibrio tenemos la ecuaci\'on ~\ref{siete}, dando como resultado la siguiente ecuaci\'on:
\begin{equation*}
K' w =\frac{ [H^+][OH^-]} {[\ce{H2O}]} = 1.86\times10^{-16} M\textrm{     a 298 K}
\end{equation*}

Si Kw es igual a:\\
Kw = [H$^+$][OH$^-$]


Despejando Kw
\begin{equation*}
Kw = K'w\cdot[\ce{H2O}] = 1.86 \times10^{-16}\cdot(0.55mol/L) = 1\times10^{-14}
\end{equation*}


De acuerdo con la ecuaci\'on ~\ref{uno}, se sabe que:
\begin{equation*}
[\ce{CO2 \cdot H2O}] = H_{\ce{CO2}} P_{\ce{CO2}}  
\end{equation*}


donde: H = 3.4$\times$10$^{-2}$$\frac{M}{atm}$


Haciendo las constantes de equilibrio para las reacciones ~\ref{dos} y ~\ref{tres}
\begin{equation*}
K_1 = \frac{[H^+][\ce{HCO3}^-]} {[\ce{CO2\cdot H2O}]} = 4.28\times10^{-7} M
\end{equation*}
\begin{equation*}
K_2 =\frac{[H^+][\ce{CO3}^{2-}]} { [\ce{HCO3}^-]} = 4.687\times10^{-11} M
\end{equation*}


Aplicando el principio de electroneutralidad tenemos que las cargas positivas deben ser igual a las negativas, entonces:
\begin{equation*}
[H^+] = [OH^-]+[\ce{HCO3}^-] +2[\ce{CO3}^{2-}]
\end{equation*}


Ahora se hace lo mismo para el caso del \ce{NH3}\\
Si la reacci\'on es :

$$\ce{NH3_{ac} <=> NH4^+ + OH^-}$$

Entonces:
\begin{equation*}
H_{\ce{NH3}} =\frac{ [\ce{NH3}_{(ac)}]} { P_{\ce{NH3}}} = 5.8\times10^1\frac{M}{atm}
\end{equation*}

Haciendo la constante de equilibrio
\begin{equation*}
K =\frac {[\ce{NH4}^+][OH^-]} {[\ce{NH3}_{(ac)}]} = 5.6\times10^{-10}
\end{equation*}

Despejando el $\ce{NH4}^+$
\begin{equation*}
\ce{NH4}^+ = \frac{K_1H_{\ce{NH3}}P_{\ce{NH3}}\cdot [H^+]} { Kw} 
\end{equation*}

Por el principio de electroneutralidad tenemos:
\begin{equation*}
[H^+] + [\ce{NH4}^+] = [OH^-] + [\ce{HCO3}^-] +2[\ce{CO3}^{2-}]
\end{equation*}



\begin{equation*}
[\ce{SO2}\cdot\ce{H2O}] = H_{\ce{SO2}} P_{\ce{SO2}}
\end{equation*} 
\begin{equation*}
H_{\ce{SO2}} =\frac{ [\ce{SO2}\cdot\ce{H2O}]} { P_{\ce{SO2}}} = 1.2\frac{M}{atm}
\end{equation*}

\begin{equation*}
K'_1 = \frac{[H^+][\ce{HSO3}^-]} {[\ce{H2SO3}]} = 1.71\times10{-2}
\end{equation*}

\begin{equation*}
K'_2 = \frac{[H^+] [SO^{2-}]}{[\ce{HSO3}^-]} = 5.98\times10^{-8}
\end{equation*}

\begin{equation*}
[H^+] = [OH^-] + [\ce{HSO3}^-] +2[\ce{SO3}^{2-}]
\end{equation*}

Despejando de K$_1$ [\ce{HSO3}$^-$]

\begin{equation*}
[\ce{HSO3}^{-}] =\frac{ K_1[\ce{H2SO3}]} {[H^+]} 
\end{equation*}

\begin{equation*}
[\ce{HSO3}^-]= \frac{K_1 H_{\ce{H2SO3}}P_{\ce{H2SO3}}} {[H^+]}
\end{equation*}

\begin{equation*}
[\ce{SO3}^{2-}] =\frac{K'_2[\ce{HSO3}^-]} {[H^+]} 
\end{equation*}

\begin{equation*}
[\ce{SO3}^{2-}] =\frac{K'_1K'_2H_{\ce{H2SO3}}P_{\ce{H2SO}}} {[H^+]}
\end{equation*}

Juntando los tres compuestos

\begin{equation*}
[H^+] + [\ce{NH4}^+] = [OH^-]+[\ce{HCO3}^-]+2[\ce{CO3}^{2-}]+[\ce{HSO3}^-]+2[\ce{SO3}^{2-}]
\end{equation*}

 \begin{equation*}
\begin{split}
[H^+] + \frac{KH_{\ce{NH3}}P_{\ce{NH3}}}{[OH^-]} &= \frac{Kw}{[H^+]} + \frac{K_1H_{\ce{CO2}}P_{\ce{CO2}}}{[H^+]} +\frac{2K_1K_2H_{\ce{CO2}}P_{\ce{CO2}}}{[H^+]^2} + \\
&\quad\frac{K'_1H_{\ce{SO2}}P_{\ce{SO2}}} {[H^+]} + \frac{2K'_1K'_2H_{\ce{SO2}}P_{\ce{SO2}}}{[H^+]^2}
\end{split}
\end{equation*}



 \begin{equation*}
\begin{split}
[H^+] + \frac{KH_{\ce{NH3}}P_{\ce{NH3}}[H^+]}{Kw} &= \frac{Kw}{[H^+]} + \frac{K_1H_{\ce{CO2}}P_{\ce{CO2}}}{[H^+]} +\frac{2K_1K_2H_{\ce{CO2}}P_{\ce{CO2}}}{[H^+]^2} +\\ &\quad\frac{K'_1H_{\ce{SO2}}P_{\ce{SO2}}} {[H^+]} + \frac{2K'_1K'_2H_{\ce{SO2}}P_{\ce{SO2}}}{[H^+]^2}
\end{split}
\end{equation*}

Multiplicando por [H$^+]^2$
 \begin{equation*}
\begin{split}
[H^+]^3 + \frac{KH_{\ce{NH3}}P_{\ce{NH3}}[H^+]^3}{Kw} &= Kw [H^+] + K_1H_{\ce{CO2}}P_{\ce{CO2}}[H^+] + 2K_1K_2H_{\ce{CO2}}P_{\ce{CO2}} +\\ &\quad K'_1H_{\ce{SO2}}P_{\ce{SO2}}[H^+] + 2K'_1K'_2H_{\ce{SO2}}P_{\ce{SO2}}
\end{split}
\end{equation*}

Factorizando [H$^+]^3$ y [H$^+$] e igualando a cero

\begin{equation*}
\begin{split}
[H^+]^3 \left( 1 + \frac{HKP_{\ce{NH3}}}{Kw}\right)-&[H^+] (Kw + H_{\ce{CO2}} K_1 P_{\ce{CO2}} + H_{\ce{SO2}}K'_1 P_{\ce{SO2}})\\ &\quad -2 H_{\ce{CO2}}K_1K_2P_{\ce{CO2}} -2H_{\ce{SO2}}K’_1K’_2P_{\ce{SO2}} = 0
\end{split}
\end{equation*}

Para resolver una ecuaci\'on de tercer grado se sabe que:\\

$x^3 + px = q$

\begin{equation*}
x = ^3\sqrt{\sqrt{\left(\frac{p}{3}\right)^3 + \left(\frac{q}{2}\right)^2}
 + \frac{q}{2}}  -  ^3\sqrt{\sqrt{ \left(\frac{p}{3}\right)^3 + \left(\frac{q}{2}\right)^2} - \frac{q}{2}}
\end{equation*}

Donde:\\
x = [H$^+$]\\\\
\begin{equation*}
p =\frac{Kw + H_{\ce{CO2}} K_1 P_{\ce{CO2}} + H_{\ce{SO2}}K'_1 P_{\ce{SO2}}}{\left( 1 + \frac{HKP_{\ce{NH3}}}{Kw}\right)}
\end{equation*}

\begin{equation*}
q =\frac{-2 H_{\ce{CO2}}K_1K_2P_{\ce{CO2}} -2H_{\ce{SO2}}K'_1K'_2P_{\ce{SO2}}}{\left( 1 + \frac{HKP_{\ce{NH3}}}{Kw}\right)}
\end{equation*}

pH = -log[H$^+$]\\
Kw = 1$\times$10$^{-14}$\\
K = 5.6$\times$10$^{-10}$\\
K$_1$ = 4.283$\times$10$^{-7}$M\\
K$_2$ = 4.687$\times$10$^{-11}$M\\
K'$_1$ = 1.71$\times$10$^{-2}$\\
K'$_2$ = 5.98$\times$10$^{-8}$\\
H = 5.8$\times$10$^{1}$ $\frac{M}{atm}$\\
H$_{\ce{SO2}}$ = 1.2$\frac{M}{atm}$\\
H$_{\ce{CO2}}$ = 3.4$\times$10$^{-2}$ $\frac{M}{atm}$\\

Variando:\\ 
P$_{\ce{CO2}}$ de 330 -- 450 ppm\\
P$_{\ce{NH3}}$ de 0 -- 4 ppb \\
P$_{\ce{SO2}}$ de 0 --150 ppb\\

Haciendo el c\'alculo para el m\'as \'acido\\
P$_{\ce{CO2}}$= 450ppm\\
P$_{\ce{NH3}}$= 0\\
P$_{\ce{SO2}}$= 150ppb\\

[H$^+$] = 5.55$\times$10$^{-5}$\\
pH = 4.26\\

Haciendo el c\'alculo para el m\'as alcalino\\
P$_{\ce{CO2}}$= 380ppm\\
P$_{\ce{NH3}}$=4 ppb \\
P$_{\ce{SO2}}$=0\\

[H$^+$] = 6.29$\times$10$^{-7}$\\
pH = 6.2\\

  
%  \reaction{CO2(g) <=> H2CO3(ac)}
%   
%  \reaction{H2CO3(ac) + H2O <=> HCO3^-{(ac)}- + H3O^+(ac)}
%  \reaction{HCO3^-(ac) + H2O <=> HCO3^{2-}{(ac)}- + H3O^+(ac)}
%  
%  \begin{equation*}
%  K_{a1}= \frac{[H_3O^+][HCO_3^-]}{[H_2CO_3_{ac}]}
%  \end{equation}
%  
%   \begin{equation*}
%   [HCO_3^-]=\frac{K_{a1}[H_2CO_3^-]}{[H_3O^+]}
%   \end{equation}
%   
%    \begin{equation*}
%    [CO_3^{2-}]=\frac{K_{a2}[HCO_3^-]}{[H_3O^+]}
%    \end{equation}
%    
%     \begin{equation*}
%     K_w= [OH^-][H_3O^+]
%     \end{equation}
%     
%      \begin{equation*}
%      [OH^-] = \frac{K_w}{[H_3O^+_{(ac)}]}
%      \end{equation}
%      
%       \begin{equation*}
%       [H_3O^+]=[HCO_3^-]+2[CO_3^{2-}_{(ac)}]+[OH^-]
%       \end{equation}
%       
%        \begin{equation*}
%        [H_3O^+] = \frac{K_{a1}HP_{CO2}}{[H_3O^+]}+\frac{2K_{a1}K_{a2}HP_{CO2}}{[H_3O^+]^2}+\frac{K_w}{[H_3O^+]}
%        \end{equation}
%        
%         \begin{equation*}
%         [H_3O^+]^3 - (K_{a1}HP_{CO2} + K_w)[H_3O^+] - 2K_{a1}K_{a2}HP_{CO2} = 0
%         \end{equation}
%         
%          \begin{equation*}
%         [H_3O^+]= [HCO_·^-]+2[CO_3^{2-}]+[OH^-]+[HSO_3^-]+2[SO_3^{2-}]
%         \end{equation}
%         
%          \begin{equation*}
%           [H_3O^+] + [NH_4^+] = [HCO_3^-] + 2[CO_3^{2-}_{(ac)}] + [OH^-] + [HSO_3^-] + 2[SO_3^{2-}]
%           \end{equation}