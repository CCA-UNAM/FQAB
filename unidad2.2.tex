\section{Equilibrio qu\'{\i}mico}
Sobre la base de la reversibilidad de las reacciones qu\'{\i}micas, se estudia y se define el concepto de equilibrio qu\'{\i}mico, haciendo \'enfasis en que se trata de un equilibrio din\'amico. Se analiza el significado de la constante de equilibrio y se aplica el principio de Le Chatelier para predecir la direcci\'on de una reacci\'on cuando \'este se altera. Se define los \'acidos y bases de acuerdo a la teor\'{\i}a de Br\"onsted-Lowry y se clasifican las bases en fuertes y d\'ebiles. Finalmente se estudia la relaci\'on entre la concentraci\'on de H+ y $p$H.
\subsection{Definici\'on}
Cualquier sistema en equilibrio representa un estado din\'amico en el que se llevan a cabo dos o m\'as procesos opuestos al mismo tiempo y a la misma rapidez. Un equilibrio qu\'{\i}mico es un sistema din\'amico en el que se efect\'uan dos o m\'as reacciones qu\'{\i}micas opuestas al mismo tiempo y a la misma rapidez. Cuando la rapidez de la reacci\'on hacia adelante es igual a la de la  reacci\'on inversa, existe un estado de \textbf{equilibrio qu\'{\i}mico}\index{equirapidezibrio qu\'{\i}mico}.  Las concentraciones de los productos y de los reactivos no cambian, y el sistema parece encontrarse en suspenso porque los productos est\'an reaccionando a la misma rapidez a la que se forman.

Una soluci\'on saturada de sal es un estado de equilibrio:
\begin{center}
\ce{NaCl_{(s)} <=>  Na+_{(ac)} + Cl_{(ac)}-}
\end{center} 
En el equilibrio, los cristales  de sal  (\ce{NaCl}) s\'olidos (s) se disuelven continuamente, y los iones de \ce{Na+} y \ce{Cl-} acuosos (ac), se cristalizan tambi\'en continuamente. Ambos procesos se llevan a cabo con la misma rapidez.
\subsection{Reversibilidad de las reacciones}
Hasta ahora hemos considerado que las reacciones qu\'{\i}micas se llevan a cabo principalmente  en donde los 
reactivos conduce a los productos. Sin embargo, muchas reacciones no proceden sino hasta su terminaci\'on por que son reversibles; esto es, cuando se forman los productos, reaccionan entre s\'{\i} produciendo los reactivos
iniciales.

Una \textbf{reacci\'on qu\'{\i}mica reversible}
\index{reaccion@reacci\'on!reversibilidad} es aquella en la que los productos que se forman reaccionan produciendo los reactivos originales. Tanto la reacci\'on hacia adelante como la inversa se llevan a cabo
simult\'aneamente.

\begin{example}
Un ejemplo es la interconversi\'on entre el di\'oxido de nitr\'ogeno (NO$_2$) y el tetr\'oxido de nitr\'ogeno (N$_2$O$_4$) es una evidencia visible de la reversibilidad de una reacci\'on. El \ce{NO_2} es un gas caf\'e rojizo que cambia, al enfriarse, a \ce{N2O4}, l\'{\i}quido amarillo que hierve a 21.2 $^\circ$C. La reacci\'on es reversible al
calentar el \ce{N2O4}.

$$\ce{2NO2_{(g)} ->[\text{Enfriamiento}] N2O4_{(l)}}$$

$$ \ce{N2O4_{(l)} ->[\text{Calentamiento}] 2NO2_{(g)}} $$

Ambas reacciones se pueden  representar en una sola con una doble flecha (\ce{<-->}), que indica que las reacciones se llevan a cabo en ambas direcciones al  mismo tiempo.

\begin{center}
\ce{ 2 NO2_{(g)}  <=> N2O4_{(l)} }
\end{center} 
\end{example}

\subsection{Constante de equilibrio}
En una reacci\'on qu\'{\i}mica reversible en equilibrio, las  concentraciones de reactivos y productos son constantes; esto es, no var\'{\i}an. La rapidez de la reacci\'on  directa y de la reacci\'on inversa es constante, y se puede escribir una expresi\'on  usando la constante de equilibrio,\index{reaccion@reacci\'on!constante de equilibrio} que relaciona a los productos con los reactivos. Para la reacci\'on general
\begin{center}
\ce{a A + bB <=> c C + d D}
\end{center}
a temperatura constante, se define la constante de reacci\'on:
\begin{equation}
K_{eq} = \frac{[C]^c[D]^d} {[A]^a [B]^b}
\end{equation}

Donde $K_{eq}$ es la constante a determinada temperatura, y su nombre es \textbf{constante de equilibrio}. Las cantidades en par\'entesis rectangulares son las concentraciones de cada substancia en moles por litro. Los exponentes a, b, c y d son los coeficientes de las substancias de la ecuaci\'on balanceada. Las unidades de $K_{eq}$ no son las mismas para todas las reacciones de equilibrio; sin embargo, generalmente se omiten las unidades.

\begin{example}

Escribir las constantes de equilibrio para:

\hskip.9in a) \ce{ 3 H2_{(g)} + N2_{(g)} <=>  2NH3_{(g)}}

\hskip.9in b) \ce{CO_{(g)} + 2H2_{(g)} <=>  CH3OH_{(g)}}

\begin{description}
\item[(a)]El coeficiente del \'unico producto, \ce{NH3_{(g)}}, es 2. Por lo tanto, el numerador de la constante de equilibrio ser\'a $[\ce{NH3}]^2$. Se encuentran presentes dos reactivos, \ce{H2}, con un coeficiente de 3, y \ce{N2}, con un coeficiente de 1. Por lo tanto, el denominador ser\'a   $[\ce{H2}]^3[\ce{N2}]$.  Obteni\'endose la siguiente expresi\'on:

$$K_{eq} = \frac{[\ce{NH3}]^2} {[\ce{H2}]^3[\ce{N2}]}$$

\item[(b)] En este caso la ecuaci\'on, tiene en el numerador  [ \ce{CH3OH}] y en el denominador $[\ce{CO}][\ce{H2}]^2$. La
expresi\'on de la constante de equilibrio es

$$K_{eq} = \frac{[\ce{CH3OH}]} {[\ce{CO}][\ce{H2}]^2}$$
\end{description}
\end{example}

La magnitud de la constante de equilibrio indica el grado en el que se llevan a cabo las reacciones hacia adelante e inversa. Cuando $K_{eq}$ es mayor que 1, indica que la cantidad de productos es mayor que la cantidad de los reactivos. Cuando $K_{eq}$ es menor que 1, la cantidad de reactivos en equilibrio es mayor que la cantidad de productos. Un valor muy grande de $K_{eq}$ indica que la reacci\'on hacia adelante se completa pr\'acticamente. Una $K_{eq}$ muy peque\~na significa que la reacci\'on inversa se completa en mayor proporci\'on; y que el equilibrio se desplaza hacia la izquierda (hacia los reactivos). A continuaci\'on se presentan dos ejemplos:
\begin{example}
De datos experimentales se obtiene la $K_{eq}$  para siguiente reacci\'on:

\begin{tabular}{lll}
\ce{ H2_{(g)} + I2_{(g)} <=>> 2 HI_{(g)}  }
&$K_{eq} = 54.8$ &a 425$^\circ C$\\

\end{tabular}
\vskip6pt

Esta $K_{eq}$ indica que en el equilibrio hay mucho m\'as
producto que reactivos.
\end{example}
\begin{example}

Para el caso del f\'osgeno \ce{COCl2} tenemos:\index{fosgeno@f\'osgeno}

\begin{tabular}{lll}
\ce{COCl_{2(g)} <<=> CO_{(g)} + Cl_{2(g)}}
&$K_{eq} = 7.6\times 10 ^{-4}$ &a 400$^\circ C$\
\end{tabular}
\vskip6pt

Esta $K_{eq}$ indica que el \ce{COCl2} es estable y que se tiene muy poca des\-composici\'on a CO y \ce{Cl2} a 400 $^\circ$C. El equilibrio est\'a muy a la izquierda.
\end{example}

\subsection{Principio de Le Chatelier}
El qu\'{\i}mico franc\'es Henri Le Chatelier (1850-1936), \index{Le Chatelier@\textbf{Le Chatelier}}enunci\'o en 1888 una generalizaci\'on sencilla pero de grandes alcances, a cerca del compor\-ta\-mien\-to de los sistemas en equilibrio. Esta generalizaci\'on, dice:
\textit{si se aplica un activante a un sistema en
equilibrio, el sistema responder\'a de tal modo que
contrarreste la activaci\'on y se restaure el equilibrio
bajo un nuevo conjunto de condiciones. }

\subsection{\'Acidos y Bases: teor\'{\i}a de Br\"onsted-Lowry}
\index{Br\"onsted-Lowry@\textbf{Br\"onsted-Lowry}}La palabra
\textit{\'acido} se deriva del lat\'{\i}n
\textit{acidus}, que significa ``agrio'', y tambi\'en se
relaciona con la palabra latina \textit{acetum}, que \index{vinagre}
significa ``vinagre''.

Algunas de las propiedades caracter\'{\i}sticas que normalmente se asocian a los \'acidos son las siguientes:
sus soluciones acuosas tiene sabor \'acido, y alteran el \index{tornasol} tornasol -- un colorante vegetal -- de color azul a rojo. Las soluciones acuosas de casi todos los \'acidos reaccionan con:
\begin{enumerate}
\item Metales como el zinc y el magnesio para producir hidr\'ogeno
gaseoso.
\item Bases para producir agua y una sal.
\item Carbonatos para producir di\'oxido de carbono.
\end{enumerate}
Estas propiedades se deben a los iones hidr\'ogeno que se liberan  en la soluci\'on acuosa de los \'acidos.

Por otro lado una \textit{base} es una substancia capaz de liberar iones hidroxilo, OH$^-$, en soluci\'on. Los hidr\'oxidos de los metales alcalinos (grupo 1A) y alcalinot\'erreos (grupo IIA), como LiOH, NaOH, KOH, \ce{Ca(OH)_2} y \ce{Ba(OH)_2}, son las bases inorg\'anicas m\'as comunes. A las soluciones acuosas de las bases\index{bases} se les llama \textit{soluciones alcalinas}, o \textit{soluciones b\'asicas}. Tienen un sabor c\'austico o amargo agudo, un tacto\index{caustico@c\'austico} resbaloso o jabonoso, la propiedad de alterar el tornasol de rojo a azul, y la capacidad de interaccionar con los \'acidos para formar una sal y agua.

En 1923, J. N. Br\"onsted (1897-1947), qu\'{\i}mico dan\'es y T. M. Lowry (1847-1936), qu\'{\i}mico ingl\'es, introdujeron la teor\'{\i}a de  Br\"onsted-Lowry de transferencia de protones, la cual afirma que un  \index{acido@\'acido} \gloss[word]{acido} es un donador (cede) de protones, y una base, un aceptador (recibiente) de protones. Algunos ejemplos de valores de $pH$ para diferentes substancias se presenta en el Cuadro~\ref{tabla8}.

\begin{table}[ht]
\begin{center}
\caption{Valores de $pH$ para diferentes substancias.}
\label{tabla8}
{\small \begin{tabular}{|cl|cl|} \hline
\textbf{\gloss[word]{ph}} & \textbf{Substancia}&\textbf{\gloss[word]{ph}} &
\textbf{Substancia}\\\hline 1 a 3 & Jugos G\'astricos (adulto)& 5.7 a 7.5 &
Saliva\\ 1.0 & \'Acido para acumulador&6.6 & Leche de vaca\\
1.1 &  �cido clorh�drico (HCl) 0.1M & 7.0& Agua pura\\
2.2 a 2.4 & Jugo de lim\'on&7.0 & Bilis\\
2.5 a 3.0 &  Refrescos de cola& 7.4&  L\'agrimas \\
2.7 a 3.0 &  Vinagre& 8.2  &Agua de mar\\
3.3 & Jugo de naranja &  9.0 & Bicarbonato (sol)\\
5.0 a 5.3 & Caf\'e & 10.5 & Leche de magnesia\\
5.5 & Pan & 13.0 & NaOH 0.1 N\\ \hline
\end{tabular}}
\end{center}
\end{table}

Observemos la siguiente reacci\'on de cloruro de hidr\'ogeno gaseoso
con agua para formar \'acido clorh\'{\i}drico:
\begin{equation}
\ce{HCl_{(g)} + H2O_{(l)} -> H3O+_{(ac)} + Cl^-_{(ac)}}
\label{ac:1}
\end{equation}

En el curso de la reacci\'on, el HCl dona, o cede, un prot\'on formando un ion \ce{Cl^-}, y el \ce{H2O} acepta, o recibe un prot\'on formando el i\'on \ce{H3O+}. As\'{\i} el \ce{HCl} es un \'acido y el \ce{H2O} es una base, seg\'un la teor\'{\i}a de Br\"onsted-Lowry.

Un ion hidr\'ogeno, H$^+$, no es m\'as que un prot\'on aislado y no existe por s\'{\i} solo en una soluci\'on acuosa. En el agua, un prot\'on se combina con una mol\'ecula polar dando un ion hidratado,
H$_3$O$^+$ [esto es, H(H$_2$O)$^+$], al que normalmente se le llama \textbf{\gloss[word]{ionhidronio}}\index{ionhidronio@i\'on hidr\'onio}. El prot\'on\index{proton@prot\'on} es atra\'{\i}do a una mol\'ecula polar de agua
formando un enlace covalente-coordinado con uno de los dos pares de electrones no compartidos.

Cuando un \'acido  dona un prot\'on, forma su base conjugada. Cuando una base acepta un prot\'on, forma el \'acido conjugado de esa base.

En la ecuaci\'on \ref{ac:1} el par conjugado \'acido-base es HCl{\footnotesize (\'acido)}--Cl$^-${\footnotesize (base)} y el H$_2$O{\footnotesize (base)}--H$_3$O$^+${\footnotesize (\'acido)}.

Los compuestos \textbf{\gloss[word]{anfotericos}} \index{anfot\'ericos} son capaces de reaccionar tanto como \'acido como base. Cuando reaccionan con un \'acido fuerte, se comportan como bases; cuando reaccionan con una base fuerte, su comportamiento es como \'acidos.

\begin{tabular}{lclclcl}
\ce{Zn(OH)2 (s)} & + &\ce{2HCl (ac)} & $\longrightarrow$ &  \ce{ZnCl2 (ac)}&+&\ce{2H2O}\\
\ce{Zn(OH)2 (s)} & + &\ce{2NaOH (ac)}&$\longrightarrow$& \ce{Na2Zn(OH)4 (ac)}
\end{tabular}
\subsubsection{Reacciones de los
\'acidos}\index{acido@\'acido!reacciones}
\begin{description}
\item[-]Con Metales generan hidr\'ogeno y una sal.\\
\'acido +  metal $ \longrightarrow $ hidr\'ogeno  +  sal\\
\begin{tabular}{lclclcl}
\ce{2HCl(ac)}& $+$&\ce{Ca(s)}&$\longrightarrow$&\ce{H2 ^} &$+$& \ce{CaCl2 (ac)}\\
\ce{H2SO4 (ac)}&$+$& \ce{Mg (s)} &$\longrightarrow$& \ce{H2 ^}&$ +$& \ce{MgSO4 (ac)}\\
\ce{4HC2H3O2 (ac)}&$ + $&\ce{2Al (s)}&$ \longrightarrow$& \ce{2H2 ^ }&$ +$&\ce{2Al(C2H3O2)2 (ac)}
\end{tabular}

El \'acido n\'{\i}trico es oxidante y reacciona con los metales
produciendo agua en vez de hidr\'ogeno:
\begin{center}
\ce{3Zn (s) + 8HNO3 (dil)  ->  3Zn(NO3)2 (ac) +2NO(g) +4H2O}
\end{center}
\item[-] Con bases\\
Es una reacci\'on de neutralizaci\'on
\index{reaccion@reacci\'on!neutralizaci\'on} en soluciones acuosas los productos
son una sal y agua. 

\'acido +  base $ \longrightarrow $ sal + agua\\
{\small  \begin{align*}
\ce{HBr (ac)  + KOH (ac)} &\ce{-> KBr (ac)  + H2O}  \\
\ce{2HNO3 (ac) + Ca(OH)2 (ac)}&\ce{ -> Ca(NO3)2 (ac) + H2O}\\
\ce{2H3PO4 (ac) + 3Ba(OH)2 (ac)}& \ce{-> Ba3(PO4)2 v + 6H2O}
\end{align*}}

\item[-] Reacci\'on con \'oxidos met\'alicos\\
\'acido +   \'oxidos met\'alicos $ \longrightarrow $ sal + agua\\

{\small \begin{align*}
\ce{2HCl (ac) + Na2O (s)} &\ce{-> 2NaCl (ac) + H2O} \\
\ce{H2SO4 (ac) + MgO (s)} &\ce{ ->  MgSO4 (a) + H2O} \\
\ce{6HCl (ac) + Fe2O3 (s)} &\ce{ -> 2Fe2Cl3 (a) + 3H2O}
\end{align*}}

\item[-] Reacci\'on con carbonatos\\
\'acido +   carbonatos $ \longrightarrow $ sal + agua + bi\'oxido de
carbono\\
{\footnotesize \begin{tabular}{lclclclcl}
\ce{2HCl (ac)} &$+$ &\ce{Na2CO3 (ac)} &$\longrightarrow$& \ce{2NaCl (ac)}&$+$ & \ce{H2O}&$+$ &  \ce{CO2 ^}\\
\ce{H2SO4 (ac)}&$+$&\ce{MgCO3 (ac)}&$\longrightarrow$&\ce{MgSO4 (ac)} & $+$ &\ce{H2O}&$+$ & \ce{CO2 ^} \\
\end{tabular}\\}
\end{description}

\subsubsection{Reacciones de las bases} \index{reaccion@reacci\'on!de las bases}
\begin{description}
\item[-]  Hidr\'oxidos anfot\'ericos capaces de reaccionar tanto como
�cidos como bases.

{\small \begin{tabular}{lclclcl}
\ce{Zn(OH)2 (s)} &$+$& \ce{2HCl (ac)} &$\longrightarrow$& \ce{ZnCl2 (ac)} &$+$& \ce{2H2O}\\
\ce{Zn(OH)2 (s)} &$+$& \ce{2NaOH (ac)} &$\longrightarrow$& \ce{Na2Zn(OH)4 (ac)}
\end{tabular}}
\item[-] Reacciones de NaOH y KOH con algunos metales. Producen una sal
e hi\-dr\'o\-ge\-no.

base + metal + agua $\longrightarrow$ sal + hidr\'ogeno

{\footnotesize 
\begin{tabular}{lclclclcl}
\ce{2NaOH (ac)} &$+$& \ce{Zn (s)} &$+$& \ce{2H2O} & $\longrightarrow$&
\ce{Na2Zn(OH)4 (ac)}&$+$&\ce{H2 ^} \\
\ce{2KOH (ac)} &$+$& \ce{2Al (s)} &$+$& \ce{6H2O} & $\longrightarrow$&
\ce{2KAl(OH)4 (ac)}&$+$&\ce{3H_2 ^}
\end{tabular}
}
\item[-] Reacciones con sales.

base + sal $\longrightarrow$ hidr\'oxido + sal\\
{\footnotesize 
\begin{tabular}{lclclcl}
\ce{2NaOH (ac)} &$+$& \ce{MnCl2 (ac)} & $\longrightarrow$&
\ce{Mn(OH)2 v}& $+$ & \ce{2NaCl (ac)}\\
\ce{3Ca(OH)2 (ac)} &$+$& \ce{2FeCl3 (ac)}& $\longrightarrow$&
\ce{2Fe(OH)3 v} &$+$& \ce{3CaCl2 (ac)}\\
\ce{2KOH (ac)} &$+$& \ce{CuSO4 (ac)} & $\longrightarrow$&
\ce{Cu(OH)2 v} &$+$&\ce{K2SO4 (ac)}
\end{tabular}
}\end{description}


\subsection{Concentraci\'on de iones H+ y $p$H}
La acidez de las soluciones que intervienen en una reacci�n qu�mica a menudo reviste importancia cr�tica, especialmente en el contexto de las reacciones bioqu�micas.  La escala de acidez por la escala de  $p$H se invent\'o para llenar la necesidad de un modo num\'erico sencillo y c\'omodo para expresar la acidez de una soluci\'on. Los valores de la escala de $p$H se obtiene mediante la conversi\'on matem\'atica de las concentraciones de iones H$^+$ mediante las siguientes expresiones:
\index{ph@$pH$}\index{acidez!ph@$pH$}
\begin{equation}
pH = \log \frac{1}{[H^+]}
\end{equation}
o bien,
\begin{equation}
 pH = -\log [H^+]
\label{ph:1}
\end{equation}
 siendo  [H$^+$] igual a la concentraci\'on de iones  H o  H$_3$O$^+$ en moles
por litro. La escala misma se basa en la concentraci\'on de iones H$^+$
en agua a 25$^\circ$C. A esta temperatura, el agua tiene una
concentraci\'on de H$^+$ igual a $1\times10^{-7}$ mol/L y se calcula de
modo que se obtiene un $p$H de 7.
$$
pH = \log \frac{1}{[H^+]}=\log \frac{1}{1\times 10^{-7}}=\log 1\times
10^7 = 7
$$

El $p$H del agua pura a 25$^\circ$C es 7 y se dice que es neutra; es
decir, ni \'acida ni b\'asica, por que las concentraciones de H$^+$ y
OH$^-$ son iguales. Las soluciones que contienen mas iones H$^+$ que
iones OH$^-$ tienen valores pH menores que 7, y las soluciones que
contienen menos iones H$^+$ que iones OH$^-$ tienen valores de $p$H
mayores que 7.\\
Cuando $[H^+] = 1\times 10^{-5}$ mol/L, $p$H = 5 (\'acido)\\
Cuando $[H^+] = 1\times 10^{-9}$ mol/L, $p$H = 9 (b\'asico)

\begin{example}[Ionizaci\'on del agua.]
\begin{center}
\ce{ 2H_2O_{(l)} <--> H_3O^+_{(ac)} + OH^-_{(ac)}}
\end{center}

La expresi\'on de la constante de equilibrio de la ionizaci\'on del agua es:

\begin{equation}
  K_{eq} = \frac{[\ce{H3O+ (ac)}][\ce{OH^- (ac)}]}{[\ce{2H2O (l)}]} 
\end{equation}

La constante de ionizaci\'on del agua a 25$^\circ$ es:
\begin{equation}
  K_w = K_{eq} [\ce{H2O}]
\end{equation}

entonces:

\begin{equation}
 K_w = [\ce{H3O^+}][\ce{OH^-}]
\end{equation}

y $ K_w = 1\times 10^{-14}$. Tenemos que  $pK_w = p[\ce{H}]  + p[\ce{OH}] $  lo que representa que  $14 = p$H$ + p$OH.
\end{example}

En el \textbf{Cuadro~\ref{ph-poh}} se muestra la relaci\'on entre el $pH$ y el $pOH$ para diferentes substancias.\index{poh@$pOH$}\index{acidez!poh@$pOH$}

\begin{table}[hbt]
\caption{Valores de $pH$ y $pOH$ para diferentes substancias}
\label{ph-poh}
\begin{center}
\begin{tabular}{l|cccc}\hline
\textbf{Substancia} & $[\ce{H^+}]$ & $pH$& $pOH$ & $[\ce{OH}]$\\ \hline
Sangre & $3.98\times 10^{-8}$&$7.4$ &$6.6$&$2.52\times 10^{-7}$\\ 
Jugo de lim\'on &$1\times10^{-2}$&$2$&$12$&$1\times10^{-12}$\\
Saliva &$3.2\times 10^{-7}$ & $6.5$ & $7.5$ &$3.1 \times 10^{-8}$\\
Refresco de Cola& $1 \times 10^{-3} $&$3$ & $11 $ &$1 \times 10^{-11} $
\\ Bilis & $1 \times 10^{-7}$&$7$&$7$&$1 \times 10^{-7}$ \\ \hline
\end{tabular}
\end{center}
\end{table}

La \gloss[word]{constantedeacidez} \index{acidez!constante $K_a$} nos indica que tan disociado se encuentra un \'acido, tambi\'en nos muestra que
tan fuerte es, en el \textbf{Cuadro~\ref{tabla9}} se muestran algunos va\-lores de  $K_a$ para diferentes substancias.

\begin{table}[ht]
\caption{Constantes de acidez $K_a$}
\label{tabla9}
\begin{center}
\begin{tabular}{ lccc } \hline
\textbf{\'Acido} &\textbf{HA} &  $A^-$   & \textbf{Ka}\\\hline
Percl\'orico & \ce{HClO_4} & \ce{ClO_4^-} & $10^{10}$\\
Clorh\'{\i}drico &  HCl  & Cl$^-$ & $10^{7}$ \\
Hidr\'onio &  H$_3$O$^+$ &  H$_2$O  & $1$\\
Fosf\'orico &H$_3$PO$_4$ &  H$_2$PO$_4^-$& $7.5\times 10^{-3}$\\
F\'ormico&  HCOOH   &  HCO$_2^-$ & $1.8 \times 10^{-4}$\\
Ion Amonio &  NH$_4^+$ &  NH$_3$ & $5.6 \times 10^{-10}$\\
Ion etil amonio &  \ce{C_2H_5NH_3^+} & \ce{C_2H_5NH_2} & $1.6 \times
10^{-11}$\\\hline
\end{tabular}
\end{center}
\end{table}
\pagebreak

 En estas reacciones de ionizaci\'on los \'acidos aparecen como reactivos y las bases como productos.
 En el \textbf{Cuadro~\ref{tabla9}} los \'acidos m\'as fuertes est\'an arriba y a la izquierda.

La ecuaci\'on que relaciona el $p$H, el $p$K$_a$ \index{pka@$p$$K_a$} y el par
\'acido d\'ebil-base conjugada se le conoce como la ecuaci\'on de
Henderson-Hasselbach:
\begin{equation}
pH = pK_a + \log
\frac{[\textrm{aceptor de hidrones(protones)}]}
{[\textrm{donador de hidrones}]}
\label{eq2.5}
\end{equation}

\begin{example}
 Calcular el pH de una mezcla de \'acido ac\'etico 0.2M y acetato de sodio 0.3M. El $pK_a$ del \'acido ac\'etico es 4.76.

A partir de la ecuaci\'on \ref{eq2.5} tenemos:
$$ pH = pK_a + \log \frac{[\textrm{acetato}]}{{[\textrm{
\'acido ac\'etico}]}}$$ Sustituyendo los valores obtenemos:

$$pH = 4.76 + \log \frac{0.3}{0.2} = 4.76 + \log(1.5) = 4.76 +
0.18=4.94$$
\end{example}

\paragraph{\gloss[Word]{buffer}}\index{buffer}Es una mezcla de un \'acido d\'ebil y su base conjugada (o una base d\'e\-bil y su \'acido conjugado) que regula el $pH$. A esta soluci\'on tambi\'en se le conoce como soluci\'on tamp\'on.
\pagebreak

\begin{exercises}
Conteste con verdadero o falso, o complete las
siguientes preguntas:

\exer Todas las reacciones qu\'{\i}micas ocurren a la misma rapidez\dotfill (\hskip.12in)\\
\exer La rapidez qu\'{\i}mica se define como el tiempo en que tardan los productos en convertirse en reactivos\dotfill(\hskip.12in )\\
\exer La teor\'{\i}a de las colisiones dice que todas las colisiones producen reacci\'on\dotfill(\hskip.12in )\\
\exer Especie intermedia que s\'olo existe durante una fracci\'on de segundo y puede dar lugar a los productos o a reactivos, es el complejo activado\dotfill(\hskip.12in )\\
\exer Barrera de energ\'{\i}a que separa el estado de los reactivos y el estado de los produc\-tos es la energ\'{\i}a de activaci\'on \dotfill (\hskip.12in)\\
\exer Una neutralizaci\'on es aquella reacci\'on en donde se combina un \'acido con una base\dotfill (\hskip.12in)\\
\exer Una soluci\'on de un \'acido y su base conjugada forman una soluci\'on tam\-p\'on o Buffer\dotfill(\hskip.12in)\\
\exer Mencionar tres factores que influyen en la rapidez de reacci\'on.
:\hrulefill ,\hrulefill y \hrulefill .
\exer  ?`C\'omo debe ser el valor de la $K_{eq}$ con respecto al uno para que el equilibrio se desplace hacia los productos?\hrulefill
\exer Escriba la $K_{eq}$ para la siguiente reacci\'on: 
 \ce{H2 + I2 -> 2HI}\\
\exer Si un sistema en equilibrio sufre una alteraci\'on el  sistema responder\'a de tal forma que compense dicha alteraci\'on es el principio de:\hrulefill
\exer Un \'acido es un donador de :\hrulefill
\exer En la siguiente reacci\'on se\~nalar qu\'e compuestos se comportan como \'acido y cu\'ales como base \ce{HCl +H2O -> H3O+ + Cl-}
\exer Un compuesto que puede reaccionar tanto como \'acido o como base se le conoce como:\hrulefill 
\exer ?`Qu\'e representa la $p$ en el t\'ermino de $pH$:\hrulefill
\vskip6pt
\exer Llenar los espacios en blanco:\\

{\centering
\vskip6pt
\begin{tabular}{l|c|c|c|c}
Substancia & $[H^+]$ & $pH$ & $pOH$ & [OH]\\ \hline
Leche de vaca &    & $6.6$ &$7.4$ &  \\
\hline Jugo de lim\'on & &$2$ & &$1\times10^{-12}$ \\
\hline 
\end{tabular}}

\exer A partir de la siguiente reacci\'on:\\
\ce{2HNO3 + Mg(OH)_2 -> Mg(NO_3)_2 + 2H_2O}\\

 ?`Qu\'e cantidad de \'acido n\'{\i}trico (\ce{HNO_3}) puede reaccionar con 200 g de hidr\'oxido de magnesio (\ce{Mg(OH)_2}). 

\end{exercises}
